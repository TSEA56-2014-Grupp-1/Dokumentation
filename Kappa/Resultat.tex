
\section{Resultat}

Projektgruppen har under projektets gång visat att en robot så som beskriven i den tekniska dokumentationen skulle vara ett fullvärdigt alternativ att användas i ett faktiskt lager. Detta har visats i de tester som utförts där roboten uppfyller alla krav med prioritet ett med god repeterbarhet. De tester som genomförts har i huvudsak bestått av att låta roboten plocka upp och sätta ner objekt medan den har följt banan. För att säkerställa robotens förmåga att konsekvent uppfylla kraven har den testats på en rad olika banor designade för att tänja på de gränser som finns uppsatta i banspecifikationen, bilaga \ref{sec:banspec}. Objekten har placerats i olika vinklar och på olika avstånd för att testa sidoskannrar och armenheten. Avbrott i tejpen, kurvradie samt korsningar har hanterats i olika situationer ett stort antal gånger.

Vidare har robotens kapacitet för att helt autonomt plocka upp och sätta ner föremål enligt de RFID-taggar som ligger utlagda vid plockstationerna. All relevant sensordata vidarebefordras även till en persondator vilket möjliggör övervakning av roboten utan behov av direkt kontakt mellan människa och robot. Att roboten är autonom gör även att den är väldigt enkel att använda eftersom den efter uppstart inte kräver någon vidare mänsklig inblandning.
