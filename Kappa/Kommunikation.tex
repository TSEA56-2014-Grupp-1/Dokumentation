

\subsection{Kommunikation}

Kommunikationsenhetens uppgift är att förmedla data mellan robot, dator och omgivningen. Detta sköts dels via den skärm som finns på roboten och dels via det blåtandsmodem som sitter monterat på roboten. Skärmen har främst använts till att felsöka roboten under utvecklingsarbetet. I den färdiga produkten har datorn istället tagit över en stor del av skärmens informationsförmedlande uppgift. Skärmen ger dock även i robotens slutgiltiga utförande information om plockstationer samt RFID-taggarnas värde.

För att styra roboten finns ett program för en persondator där olika kommandon kan skickas till roboten. Från denna kan hjulen styras med antingen musklick eller tangentbord och det går även att starta eller stoppa linjeföljning och sätta nya parametrar till regleringen. Vidare finns möjligheten att styra armen, antingen manuellt eller genom att aktivera förprogrammerade rörelser.

I datorgränssnittet presenteras även information som skickas från roboten. De styrbeslut som roboten tar under autonom körning presenteras i gränssnittets loggfönster och den från linjesensorn beräknade tyngdpunkten plottas i en realtidsgraf. Vidare presenteras de avstånd som fås in från sidoskannrarna i ytterligare en graf. Datan presenteras endast då den är relevant. Detta innebär att man exempelvis inte får någon graf över tyngdpunkten då roboten står vid en plockstation och inte heller någon data från sidoskannrarna då roboten rör sig mellan plockstationer.




