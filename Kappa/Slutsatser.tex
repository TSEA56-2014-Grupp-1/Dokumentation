
\section{Slutsatser}

Gruppen har under projektet dragit stor nytta av det grundliga arbete som gjordes i designspecifikationen. Även om arbetet med designspecifikationen tog ansenliga mängder tid i anspråk har det sett över hela projektet lönat sig. Detta då hela gruppen redan på ett tidigt stadie i projektet haft en mycket god förståelse för hur roboten ska fungera. Då robotens delsystem senare under projektet skulle integreras med varandra var detta en stor fördel eftersom att alla medlemmar i gruppen haft en bra uppfattning av resterande medlemmars arbetsuppgifter. Mycket tack vare detta har utvecklingsarbetet till stor del kunnat utföras parallellt på de olika delsystemen. 

Det gruppen är mest nöjda med vad det gäller den tekniska lösningen är den autonoma upplockningen. Roboten klarar av att konsekvent plocka upp föremål utan mänsklig inblandning samt att sätta ner dem utan att de välts eller tappas. Detta var från början inte något som sågs som en självklar funktion hos roboten och vidare har ingen av de andra två projektgrupperna med samma uppgift lyckats med detta.

Om mer tid gavs skulle vi främst vilja effektivisera funktionaliteten vi redan har. Något utmanande vore också att få roboten att interagera med liknade robotar i en lagermiljö. En trippmätare hade med fördel kunnat implementeras så att roboten alltid stannar exakt ovanför stationstejpen, oavsett vilken hastighet roboten håller. Beslut om ifall roboten tjänar på att vända om och köra tillbaka istället för att fortsätta framåt hade också varit en fördelaktig funktion för att ytterliggare effektivisera tiden det tar för roboten att utföra sitt uppdrag.

Git och Github hade kunnat skötas snyggare och med tydligare indelning. Detta hade man kunnat löst genom att ha tagit med det tänkta upplägget i designspecifikationen och genom att se till att alla gruppens medlemmar varit mer insatta i hur Github fungerar.
Om gruppen fick göra om samma uppdrag igen hade det antagligen blivit samma eller en mycket lik produkt med liknande implementationer som de som gjorts i det här projektet. Överlag tycks det svårt att utnyttja komponenterna på ett sådant sätt att roboten hade blivit avsevärt mycket effektivare i utförande av sitt uppdrag utöver implementation av de funktioner som hade gjorts om gruppen haft mer tid på sig. Viss funktionalitet hade också implementerats på ett mer generellt sätt, vilket hade kunnat effektivisera arbetet, men det hade med största sannolikhet lett till en liknande slutprodukt.
