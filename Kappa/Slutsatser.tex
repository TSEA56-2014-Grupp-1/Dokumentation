
\section{Slutsatser}

Gruppen har under projektet dragit stor nytta av det grundliga arbete som gjordes i designspecifikationen. Även om arbetet med designspecifikationen tog ansenliga mängder tid i anspråk har det, sett över hela projektet, lönat sig. Detta då hela gruppen redan på ett tidigt skede i projektet haft en mycket god förståelse för hur roboten ska fungera. Då robotens delsystem senare under projektet skulle integreras med varandra var detta en stor fördel eftersom att alla medlemmar i gruppen haft en bra uppfattning av resterande medlemmars arbetsuppgifter. Mycket tack vare detta har utvecklingsarbetet till stor del kunnat utföras parallellt på de olika delsystemen. 

Det gruppen är mest nöjd med, vad det gäller den tekniska lösningen, är den autonoma upplockningen. Roboten klarar av att konsekvent plocka upp föremål utan mänsklig inblandning samt att sätta ner dem utan att de välts eller tappas. Detta var från början inte något som sågs som en självklar funktion hos roboten och vidare har ingen av de andra två projektgrupperna med samma uppgift lyckats med detta.

Om mer tid gavs skulle gruppen främst vilja effektivisera funktionaliteten som redan finns. Något som vore utmanande är också att få roboten att interagera med liknade robotar i en lagermiljö. En trippmätare hade med fördel kunnat implementeras så att roboten alltid stannar exakt ovanför RFID-taggen, oavsett vilken hastighet roboten håller. Beslut om ifall roboten tjänar på att vända om och köra tillbaka istället för att fortsätta framåt hade också varit en fördelaktig funktion för att ytterligare effektivisera tiden det tar för roboten att utföra sitt uppdrag.

Versionshanteringen var i projektets slutskede oorganiserad på grund av bristande förkunskaper om hur versionshanteringsverktyget Git fungerade. Detta hade kunnat lösas genom att beskriva det tänkta upplägget i designspecifikationen och genom att se till att alla gruppens medlemmar varit mer insatta i hur Git skulle användas.

Om gruppen fick göra om samma uppdrag igen hade det antagligen blivit samma, eller en mycket lik produkt, med liknande lösningar som använts i detta projekt. Med samma komponenter bedöms det vara svårt att nå ett bättre resultat än det som uppnåtts. Viss funktionalitet hade implementerats på ett mer generellt sätt, vilket hade kunnat effektivisera arbetet, men det hade med största sannolikhet lett till en liknande slutprodukt.
