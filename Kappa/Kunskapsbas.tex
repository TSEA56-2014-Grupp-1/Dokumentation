


\section{Kunskapsbas}

Majoriteten av komponenterna som använts i projektet finns dokumenterade med datablad och förklaringar på ISY:s (Institutionen för systemteknik vid Linköpings universitet) databladssida Vanheden \footnote{\url{https://docs.isy.liu.se/twiki/bin/view/VanHeden}} och det är i huvudsak därifrån som specifikationer och data har inhämtats. Där finns bland annat datablad för mikroprocessorn ATmega 1284P, som används för alla fyra delsystem i projektet. 

Kunskapen som varit nödvändig för att genomföra projektet har till största del tillkommit projektgruppen under deras tidigare studier där gruppmedlemmarna bland annat har lärt sig programmering och enklare elektronikkonstruktion. Vidare har de primära källorna för ytterligare information varit de datablad som finns på Vanheden, samt projektgruppens handledare vid ISY. 

I övrigt har de dokument som skapats i förstudien och uppstarten av projektet använts som stöd under det fortsatta arbetet. Projektets dokument beskrivs i avsnitt \ref{sec:utforande}. Särskilt har designspecifikationen legat till grund för allt det konstruktions- och programmeringsarbete som skett i projektet.