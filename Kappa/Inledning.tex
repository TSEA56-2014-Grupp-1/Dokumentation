

\section{Inledning}
Sju studenter vid Linköpings Universitet valde i deras kandidatprojektkurs TSEA56 att konstruera en automatiserad lagerrobot utifrån beställarens krav på design och funktion. Projektet gjordes med avsikten att ge studenterna bildning och erfarenhet att med de kunskaper erhållna från tidigare kurser kunna genomföra ett omfattande uppdrag med samma metodik som förekommer i arbetslivet.

Produkten är en lagerrobot på fyra hjul med en robotarm monterad ovanpå. En linjesensor framtill används för att roboten ska kunna följa en linje bestående av svart tejp, detektera plock- och avlämningsstationer, hantera avbrott i tejpen samt hantera korsningar. Vid plockstationer söker roboten efter föremål med hjälp av en sidoskanner. Vid objektidentifiering beräknas en koordinat som armen använder sig av för att med hjälp av inverterad kinematik förflytta sig till föremålet autonomt. Roboten kan även styras manuellt i ett datorgränssnitt via blåtandskommunikation. 

Projektgruppen fick robotikhårdvaran (hjulbasen med strömförsörjning och robotarmen) färdigbyggd från beställaren. Projektuppgiften bestod i att designa och utveckla kretskort med mikroprocessorer, elektronik och sensorer för att klara av ett givet uppdrag. 