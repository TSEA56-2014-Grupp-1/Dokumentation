

\section{Inledning}
Sju studenter vid Linköpings universitet fick som uppgift i deras kandidatprojekt inom elektronik att konstruera en autonom robot. Projektet gjordes med avsikten att ge studenterna bildning och erfarenhet av att genomföra ett omfattande uppdrag med samma metodik som förekommer i arbetslivet.

Projektets mål var att utveckla en robot som är tänkt att arbeta i ett verkligt lager, en s.k. lagerrobot. Roboten skulle kunna röra sig runt på egen hand i ett lagerutrymme och flytta föremål mellan förutbestämda platser.

Projektgruppen fick den grundläggande hårdvaran (basen med hjul och strömförsörjning samt robotarmen) färdigbyggd från projektets beställare. Projektuppgiften bestod av att designa och utveckla kretskort med mikroprocessorer, elektronik och sensorer för att uppfylla beställarens projektdirektiv.