

\section{Inledning}
Sju studenter vid Linköpings universitet fick som uppgift i deras kandidatprojekt inom elektronik att konstruera en autonom robot. Avsikten med projektet var att ge studenterna bildning i, och erfarenhet av, att genomföra ett omfattande uppdrag med samma metodik som förekommer i arbetslivet.

Projektets mål var att utveckla en robot som är tänkt att arbeta i ett verkligt lager, en s.k. lagerrobot. Roboten skulle kunna röra sig runt på egen hand i ett lagerutrymme och flytta föremål mellan förutbestämda platser.

\nyBild{introbild.pdf}{Den färdiga roboten håller på att plocka upp ett föremål i ett ''lagerutrymme''.}{intro}{0.8}

I figur \ref{fig:intro} visas den färdiga produkten i färd med att plocka upp ett föremål. Projektgruppen fick den grundläggande hårdvaran (basplattan med hjul och strömförsörjning samt robotarmen) färdigbyggd från projektets beställare. Projektuppgiften bestod av att designa och utveckla kretskort med mikroprocessorer, elektronik och sensorer för att uppfylla beställarens projektdirektiv.

Detta dokument redogör för projektets genomförande och resultat, samt för ett antal fördjupningsuppgifter som har utförts. Alla projektdokument som har skapats har bifogats som bilagor och refereras till där det är lämpligt.