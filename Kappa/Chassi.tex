
\subsection{Chassi}

Chassienheten är den enhet som styr över de andra enheterna och bestämmer när de andra enheterna ska utföra sina uppgifter. Chassit är också den enhet som styr motorerna till hjulen. Motorerna är kopplade så att både fram och bakhjul på vardera sida får samma styrsignal, så styrningen blir liknande den på en bandvagn. Motorerna styrs genom två PWM-signaler som skickas från chassienhetens processor, se bilaga \ref{sec:tekdok}. Till chassienheten finns även en omkopplare där användaren kan välja mellan att roboten ska följa linjen autonomt eller om den bara ska gå att styra via persondatorn. Det finns även en tryckknapp som används för att påbörja autonom drift med linjeföljning.

Chassits huvudprogram startas genom ett knapptryck eller via ett kommando från persondatorn. Då beordrar chassit sensorenheten att börja söka efter en linje att följa. Ifall omkopplaren står på manuellt läge händer ingenting mer. Annars ber chassit sensorenheten med jämna mellanrum om en tyngdpunkt samt information om ifall roboten har en station till höger eller vänster. Detta görs fram tills dess att sensorenheten skickar information om att roboten är på en station, figur \ref{fig:huvudprogram} visar ett flödesschema över chassits program. Med hjälp av tyngdpunkten och PD-reglering beräknar chassit vilken hastighet hjulen på vardera sida ska ha för att det ska bli en mjuk linjeföljning, se bilaga \ref{sec:tekdok}. 

\nyBild{chassi_huvudprogram.pdf}{Översiktlig bild av Chassits huvudprogram}{huvudprogram}{0.7}


När sensorenheten informerar chassit om att den har upptäckt en station stannar roboten. Därefter beordrar chassit sensorenheten att läsa in RFID-taggen som ska ligga under roboten. Sensorenheten skickar tillbaka RFID-taggens identifikationsnummer. Ifall ingen tagg hittas kör roboten framåt och ber sensorenheten att göra en ny läsning. Hittas ingen tagg efter att roboten kört både framåt och bakåt en kort sträcka ger roboten upp och kör vidare till nästa station. Om roboten däremot hittar en tagg tar chassienheten beslut om vad som ska göras beroende på vilket ID taggen har. Beslutet beror på ifall det är en upplockningsstation eller avlämningsstation samt ifall roboten redan bär på något eller ifall stationen redan är avklarad. 

Om roboten har stannat på en station där ett föremål ska plockas upp skickar chassit ett kommando till armen som talar om att ett objekt ska plockas upp och på vilken sida av roboten föremålet befinner sig. När armen utfört sin uppgift enligt kommandot från chassit meddelar armen chassit att den är klar och ifall den klarade av att plocka upp eller lägga ner objektet. 

Chassit håller reda på huruvida alla stationer är avklarade genom att räkna hur många stationer den passerar innan den når samma station igen. När alla stationer är avklarade står roboten kvar på sista stationen och LCD-skärmen förkunnar att uppdraget är slutfört. Annars kör roboten vidare till nästa station. Vid varje beslutspunkt i chassits huvudprogram skickar chassit sitt beslut till kommunikationsenheten som i sin tur vidarebefordrar detta till den fjärranslutna datorn. Ett beslut tas och vidarebefordras vid följande beslutspunkter: 
\begin{itemize}
\item Roboten har åkt över en plockstationsmarkering
\item Roboten har läst in en RFID-tagg
\item Armen är klar med ett uppdrag, upplockning eller avlämning
\end{itemize}
