

\section{Fördjupningsuppgifter}

Som en del i kandidatprojektskursen utförde projektgruppen mer individuella fördjupningsuppgifter inom tre olika områden som är relevanta för projektet. Två gruppmedlemmar skrev en fördjupning om robotens sensorer (bilaga \ref{sec:sensoruppgift}), två skrev en uppgift om reglering av en linjeföljande robot och reglering av en robotarm (bilaga \ref{sec:regleruppgift}), och tre skrev om litiumjonbatterier och om servomotorer (bilaga \ref{sec:batteriservouppgift}). 

\subsection{Sensoruppgiften}

I den skrivuppgift som gjordes om sensorer (se bilaga \ref{sec:sensoruppgift}) utreddes de olika sensorer som används till roboten. Huvudsyftet var att ta reda på om de skulle komma att vara tillräckliga för att roboten på ett tillfredsställande sätt skulle kunna klara av uppdraget. Vidare utreddes vilka alternativa sensorer som skulle kunnat användas för att få ökad prestanda med avseende på bland annat detektion av objekt. 

Den primära slutsats som drogs i denna uppgift var att de val av sensorer som gjorts i designspecifikationen skulle komma att vara fullgoda för robotens förmåga att uppfylla de krav som satts i kravspecifikationen. De delar som man skulle tjäna mest på att byta ut är de servon och avståndssensorer som utgör sidoscannrarna. Mer högupplösta servon och avståndssensorer som inte sprider ljuset lika mycket skulle ge möjligheten till att mer precist detektera föremål. Detta i sin tur ger mer exakta koordinater till armen som då skulle kunna plocka upp föremål med högre precision.

\subsection{Servomotorer}
I den skrivuppgift som behandlade servomotorer (se bilaga \ref{sec:batteriservouppgift}) undersöktes och redovisades hur ett servo fungerar, hur man kan få ett servo att röra sig utan kraftiga ryck och hur man styr servon av typen Dynamixel AX-12A som används i projektets robotarm. Slutsatserna som drogs av undersökningen gav att ett servo har inbyggd återkoppling vilket gör att den alltid kommer att ställa in sig på önskad vinkel, oberoende av yttre störningar. Det framgick även att  Dynamixel AX-12A har inbyggt stöd för att kunna röra sig utan kraftiga ryck och att om denna funktion inte funnits hade det varit mycket svårt att implementera detta på egen hand. 

\subsection{Litiumjonbatterier}

I uppgiften om batterier (se bilaga \ref{sec:batteriservouppgift}) presenterades hur litiumjonbatterier fungerar och vilka risker som finns vid laddning, transport och användning. Det presenterades även vilka faktorer som påverkar ett batteris livslängd negativt. Dessa är främst för stor urladdning, hög temperatur och hur laddningsströmmen regleras under laddning. För att minimera slitage ska batterierna övervakas under laddning så att spänningen inte blir för hög, samt bör laddningsströmmen begränsas till en låg ström under de första och sista 10\% av laddningscykeln.

Vidare redogörs kort för en del av den forskning som bedrivs om litiumjonbatterier och alternativ till tekniken.

\subsection{Linjeföljning}

I skrivuppgiften om reglering (se bilaga \ref{sec:regleruppgift}) presenterades och testades en vanligt förekommande regleringsmetod, PID-reglering. När man ska reglera insvängning för att automatiskt följa en linje så räcker det inte att styra endast med avseende på hur stor avvikelsen är från mittpositionen för roboten. Detta på grund av att det väldigt lätt kan uppstå oscillerande rörelser och i vissa fall stationära fel, som oftast slutar med att roboten hamnar helt ur kurs. Som verktyg för att lösa detta används PID-reglering för att styra roboten.

I dokumentet väljs PD-reglering ut som den bäst lämpade reglertypen för linjeföljning, och parametrar för regleringen testas och redovisas. För att ta fram parametrar till reglerare hänvisas det till tester som gjorts på vår robot och som visat sig vara lämpliga.

\subsection{Inverterad kinematik}
I fördjupningsuppgiften inom inverterad kinematik (se bilaga \ref{sec:regleruppgift}) beräknades en modell för en godtycklig treledad arm med roterande basplatta. Olika strategier för att genomföra inverterad kinematik diskuterades. Den valda lösningen var analytisk och visades matematiskt. En modell av armen i två dimensioner testades sedan i datorn för att se att modellen fungerade tillfredsställande.

Modellen i fördjupningsuppgiften implementerades på roboten och visade sig tillräckligt exakt för att autonomt kunna plocka upp föremål från givna positioner.