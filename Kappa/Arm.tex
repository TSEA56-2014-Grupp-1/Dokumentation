\subsection{Arm}

Armen på roboten är av modell PhantomX Reactor från Trossen Robotics, med 7st AX-12A servon och 4st rörliga leder samt en gripklo. Servona skickar och tar emot data pararellt på samma kabel vilket gjorde att en tri-state buffer (fotnot: en krets som kan dela en kabel till en in- och en utport beroende på värdet på en enableport) varit nödvändig för programmet att kunna välja om den vill skicka till eller ta emot från servona. Servona kommunicerar med processorn genom USART. Varje servo har tar emot all data som skickas från processorn, men agerar endast på de kommandon som är riktade till dess eget unika ID.

Då kommandona som skickas är flera byte långa byggs dessa upp på förhand innan de skickas. Det är viktigt att det inte blir avbrott i kommunikationen för att processorn ska beräkna nästa byte. Om det inträffar skulle datan feltolkas och inte nå fram. Därför inaktiveras avbrott under själva överföringen. Vid varje sändning som riktas till ett enskilt servo fås ett svar tillbaka från med statuskod och eventuella parametrar. Detta svar används för eventuell felhantering. Eftersom gruppen valt att kunna styra robotarmen autonomt är det extra viktigt att kontrollera att varje instruktion blivit korrekt mottagen. Därför väntar programmet på svar från servona efter varje sändning gjorts. Dessa statuskoder kan sedan tolkas av programmet för att avgöra om kommandot nådde servot.

För ytterligare säkerhet gällande kommandon till servon, speciellt rörelser, har gruppen valt att endast sköta instruktioner genom funktionen reg-write som servona har. Med reg-write kan man ge ett servo en instruktion som endast utförs då ytterligare ett kommando, action-kommando, skickas till servot. Då all statusinformation finns tillgänglig kan programmet säkerställa att servona kommer agera som planerat innan rörelser utförs. Detta är väldigt användbart i de fall hela armen ska röras eftersom en led som inte rör sig kan ge en slutposition som i värsta fall skadar roboten.
Vid både manuell och autonom styrning av armen beräknas ledvinklar med inverterad kinematik (Metoden för inverterad kinematik beskrivs i bilaga \ref{sec:regleruppgift}). Armen kan styras manuellt i djupled, höjdled och rotation av basplatta. Gripklon kan öppnas och stängas manuellt. När armen rör sig manuellt räknas startpositionen ut och sedan adderas kontinuerligt önskad förändring så länge som armen förflyttas. Vid autonom rörelse får armenheten en vinkel och ett avstånd från basplattan till föremålet som ska plockas upp. Genom inverterad kinematik ställer armen in sig och greppar föremålet. Koordinaten som föremålet plockades upp på sparas ned och används vid avlämning för att garantera att föremålet inte ska falla omkull.

Huvudprogrammet för armenheten är passivt tills enheten anropas över bussen. Busskommandona sätter olika flaggor som huvudprogrammet kontinuerligt läser av.
