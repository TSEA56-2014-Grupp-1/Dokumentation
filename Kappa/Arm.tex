\subsection{Arm}

Armen på roboten är av modell PhantomX Reactor från Trossen Robotics, med sju AX-12A servon i fyra rörliga leder samt en gripklo. Servona kommunicerar med processorn över en gemensam seriell buss. Varje servo tar emot all data som skickas från processorn, men agerar endast på de kommandon som är riktade till dess eget unika ID.

Vid varje sändning som riktas till ett enskilt servo fås ett svar tillbaka med en statuskod och eventuella parametrar. Detta svar används för eventuell felhantering. Eftersom gruppen valt att kunna styra robotarmen autonomt är det extra viktigt att kontrollera att varje instruktion blivit korrekt mottagen. Därför väntar armenheten på svar från servona efter varje sändning gjorts. Dessa statuskoder kan sedan tolkas för att avgöra om kommandot nådde servot.

För ytterligare säkerhet gällande kommandon till servon, speciellt rörelser, har gruppen valt att endast sköta instruktioner genom kommandotypen ''reg-write'' som servona har. Med ''reg-write'' kan det skickas en instruktion till ett servo som endast utförs då kommandot ''action'' skickas till servot. Då all statusinformation finns tillgänglig kan programmet säkerställa att servona kommer agera som planerat innan rörelser utförs. Detta är väldigt användbart i de fall hela armen ska röras eftersom en led som inte rör sig kan ge en slutposition som i värsta fall skadar roboten.

Vid både manuell och autonom styrning av armen beräknas ledvinklar utifrån en, för armens klo, given koordinat med inverterad kinematik (metoden för inverterad kinematik beskrivs i bilaga \ref{sec:regleruppgift}). Armen kan styras manuellt i djupled, höjdled och rotation av basplatta. Gripklon kan öppnas och stängas manuellt. När armen rör sig manuellt räknas startpositionen ut och sedan adderas kontinuerligt önskad förändring så länge som armen förflyttas. Vid autonom rörelse får armenheten en vinkel och ett avstånd från basplattan till föremålet som ska plockas upp, och genom inverterad kinematik ställer armen in sig och greppar föremålet. Koordinaten som föremålet plockades upp på sparas ned och används vid avlämning för att garantera att föremålet inte ska falla omkull.

