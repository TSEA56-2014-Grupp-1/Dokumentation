

\section{Genomförande}

Projektet har genomförts enligt projektmodellen LIPS\footnote{Lätt Interaktiv Projektstyrning, en projektmodell vid Linköpings Universitet. I modellen genomförs ett projekt i tre faser: en förstudie, utförande, och en efterstudie. En projektgrupp har en tydlig projektledare, och jobbar mot en beställare som vid olika förutbestämda milstolpar tar beslut om projektets fortsatta genomförande.}. Projektets ursprung var ett projektdirektiv från beställaren där en översiktlig beskrivning av roboten  kravbilden gavs. Utifrån projektdirektivet inleddes projektet med att ta fram en tydlig kravspecifikation som skulle specificera exakt vad slutprodukten skulle bestå av.

Den projekttypen som skulle genomföras var ny för i år och har inte genomförts av tidigare studenter, projektgruppen fick således frihet att själva kunna experimentera med tankar och idéer för vad projektet skulle innefatta. Utifrån projektdirektivet tog gruppen fram ett antal krav på design och funktionalitet och rangordnade dem efter tre olika prioritetsnivåer, baserat på hur viktiga de bedömdes vara för att projektdirektivet skulle uppfyllas. Prioritet 1 innebar att kravet var tvunget att uppfyllas, prioritet 2 innebar att kravet borde uppfyllas i mån av tid, och prioritet 3 innebar att kravet skulle ses som ett förslag till framtida utbyggnad och förbättring. För mer detaljer, se kravspecifikationen i bilaga \ref{sec:kravspec}.

När kravspecifikationen väl upprättats i samråd med beställaren påbörjades planeringen av projektets genomförande. En systemskiss skrevs som gav en övergripande bild av vad som krävdes utav roboten rent tekniskt. Den var upplagd som en sorts förstudie som beskrev olika möjliga designval och tekniska lösningar, för att samla in tankar och idéer på hur projektet skulle kunna genomföras, se bilaga \ref{sec:systemskiss}. Utifrån systemskissen upprättade man en projektplan och specificerade vilka aktiviteter som skulle genomföras i projektet. Alla aktiviteter organiserades i ett beroendeträd, som visar hur olika akviteter beror på att andra aktiviteter är genomförda, för att ligga som grund för planering av när de skulle genomföras. Projektplanen är bifogad i bilaga \ref{sec:projektplan}. Till projektplanen upprättades också en tidplan för alla aktiviteter, som uppdaterades med jämna mellanrum under projektets gång. Denna specificerade noggrannt vem som skulle utföra vilken aktivitet när och hur lång tid den beräknades ta, se bilaga \ref{sec:tidplan}.

Efter att projektplanen hade godkänts av beställaren och alla projektmedlemmar påbörjades själva utförandefasen av projektet med att skriva en designspecifikation. Designspecifikationen skulle ge en detaljerad bild av hur produkten skulle fungera och se ut, tillräckligt heltäckande och detaljerad för att kunna bygga hela produkten utifrån endast denna. Detta innebar att alla designbeslut både vad gäller hårdvara och mjukvara togs i arbetet med designspecifikationen på en sådan detaljnivå att konstruktion och programmering genast kunde påbörjas då detta dokument var färdigställt. Projektgruppen utgick från systemskissen och gjorde i princip en ytterligare fördjupning och förtydligande av den. Designspecifikationen återfinns i bilaga \ref{sec:designspec}.

Då designspecifikationen färdigställts påbörjades arbetet med att bygga roboten och skriva mjukvara. Detta arbete utfördes till stor del parallellt då arbetet bedrevs i grupper om två som utvecklade olika delsystem. Det första som gjordes var att etablera en grundläggande funktionalitet genom att direkt konstruera robotens kretskort, designa och programmera busskommunikation och funktioner för att mata ut information på robotens LCD-skärm. Parallellt med detta påbörjades inläsningen av sensordata och grundläggande kontroll av robotens hjul. Arbetet gjordes i den ordning som lagts fram i tidplanen, som tagit hänsyn till aktiviteternas inbördes beroenden, för att göra arbetet så effektivt som möjligt och minimera väntetid. 

När man skriver ett mjukvaruprojekt av denna storlek (~10 000 rader kod) är det viktigt att källkoden är organiserad och versionshanterad på ett bra sätt. I projektet har koden versionhanterats från första början med verktyget Git. Hemsidan GitHub(fotnot med beskrivning och url) har använts för att enkelt samordna arbetet mellan gruppmedlemmarna, och för att underlätta spårning av problem och visualisera versionshistoriken.

Mjukvaran har i huvudsak utvecklats med utvecklingsmiljön Atmel Studio utgiven av AVR-processorernas tillverkare Atmel, för den mjukvara som ligger på roboten, och utvecklingsmiljön QtCreator för persondatorns mjukvara.