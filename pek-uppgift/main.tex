\documentclass[a4paper,12pt]{article}
\usepackage{graphicx}

\usepackage{epstopdf}
\usepackage{gensymb}
\usepackage{float}
\usepackage{mathtools}
\usepackage{setspace}
\usepackage{tabularx}
\title{Fördjupningsuppgift i kommunikation och konstruktion}
%% Definitioner för LIPS-dokument

\usepackage[swedish]{babel}
\usepackage[utf8]{inputenc}
\usepackage[T1]{fontenc}
\usepackage{times}
\usepackage{ifthen}
\usepackage[labelfont=it]{caption}

\usepackage[margin=25mm]{geometry}

\def\arraystretch{1.6}

\usepackage{fancyhdr}
\pagestyle{fancy}
\lhead{}
\chead{\LIPSprojekttitel}
\rhead{\LIPSdatum}
\lfoot{\LIPSkursnamn \\ \LIPSdokumentansvarig}
\rfoot{\LIPSprojektgrupp \\ \LIPSgruppepost}

\setlength{\parindent}{0pt}
\setlength{\parskip}{1ex plus 0.5ex minus 0.2ex}


\newcommand{\twodigit}[1]{\ifthenelse{#1<10}{0}{}{#1}}
\newcommand{\dagensdatum}{\number\year-\twodigit{\number\month}-\twodigit{\number\day}}

%% ------------------------------------------
% NYBILD
% Skapar centrerad bild med caption
%   
% #1: Filens url relativt '/bilder/'
% #2:  Caption
% #3: Label
% #4: Skalning i förhållande till textwidth
%% ------------------------------------------
\newcommand{\nyBild}[4] 
{\begin{figure}[H]
  \centering
 \emph{\includegraphics[angle=0,width=#4\textwidth]{bilder/#1}}
  \caption{\emph{#2}}
  \label{fig:#3}
\end{figure}}

%%  Redefinitions of commands containing @
\makeatletter
\makeatother

\newcommand{\LIPStitelsida}{%
{\ }\vspace{45mm}
\begin{center}
  \textbf{\Huge {\sffamily \LIPSdokumenttyp}}
\end{center}
\begin{center}
  {\Large \LIPSredaktor}
\end{center}
%\begin{center}
%  {\Large Version \LIPSversion}
%\end{center}
\vspace{60mm}
%\begin{center}
%  {\large Status}\\[1.5ex]
%  \begin{tabular}{|*{3}{p{40mm}|}}
%    \hline
%    Granskad & \LIPSgranskare & \LIPSgranskatdatum \\
%    \hline
%    Godkänd & \LIPSgodkannare & \LIPSgodkantdatum \\
%    \hline
%  \end{tabular}
%\end{center}
\newpage
}


\newenvironment{LIPSprojektidentitet}{%
{\ }\vspace{45mm}
\begin{center}
  {\Large PROJEKTIDENTITET}\\[0.5ex]
  {\small
  \LIPSprojektgrupp, \LIPSartaltermin, \LIPSprojekttitel\\
  Tekniska högskolan vid Linköpings universitet, ISY
  }
\end{center}
\begin{center}
  \begin{tabular}{|l|p{45mm}|p{25mm}|l|}
    \hline
    \textbf{Namn} & \textbf{Ansvar} & \textbf{Telefon} & \textbf{E-post} \\
    \hline
}
{
    \hline
  \end{tabular}
\end{center}
\begin{center}
  {\small
    \textbf{E-postlista för hela gruppen}: \LIPSgruppepost\\
    \textbf{Kontaktperson hos kund}: \LIPSkundkontakt\\
    \textbf{Kursansvarig}: \LIPSkursansvarig\\
    \textbf{Handledare}: \LIPShandledare\\
  }
\end{center}
\newpage
}
\newcommand{\LIPSgruppmedlem}[4]{\hline {#1} & {#2} & {#3} & {#4} \\}



\newenvironment{LIPSdokumenthistorik}{%
\begin{center}
  Dokumenthistorik\\[1ex]
  \begin{small}
    \begin{tabular}{|l|l|p{60mm}|l|l|}
      \hline
      \textbf{Version} & \textbf{Datum} & \textbf{Utförda förändringar} & \textbf{Utförda av} & \textbf{Granskad} \\
      }%
    {%
      \hline
    \end{tabular}
  \end{small}
\end{center}
}
\newcommand{\LIPSversionsinfo}[5]{\hline {#1} & {#2} & {#3} & {#4} & {#5} \\}



\newenvironment{packed_itemize}{
\begin{itemize}
	\setlength{\itemsep}{1pt}
    \setlength{\parskip}{0pt}
    \setlength{\parsep}{0pt}
}{\end{itemize}}

\newenvironment{packed_enumerate}{
\begin{enumerate}
	\setlength{\itemsep}{1pt}
    \setlength{\parskip}{0pt}
    \setlength{\parsep}{0pt}
}{\end{enumerate}}





%%% Local Variables: 
%%% mode: latex
%%% TeX-master: "kravspec_mall"
%%% End:


\usepackage{sectsty}
\allsectionsfont{\sffamily}

\frenchspacing

\renewcommand{\thepage}{\roman{page}}

\newcommand{\LIPSartaltermin}{VT14}
\newcommand{\LIPSkursnamn}{TSEA56 Elektronik kandidatprojekt}

\newcommand{\LIPSprojekttitel}{Lagerrobot}

\newcommand{\LIPSprojektgrupp}{Grupp 1}
\newcommand{\LIPSgruppepost}{tsea56-2014-grupp-1@googlegroups.com}
\newcommand{\LIPSdokumentansvarig}{LIPS Designspecifikation}

\newcommand{\LIPSkund}{ISY, Linköpings universitet, 581\,83 Linköping}
\newcommand{\LIPSkundkontakt}{Tomas Svensson, 013-28 13 68, tomass@isy.liu.se}
\newcommand{\LIPSkursansvarig}{Tomas Svensson, 013-28 13 68, 3B:528, tomass@isy.liu.se}
\newcommand{\LIPShandledare}{Anders Nilsson, 3B:512, 013-28 26 35, anders.p.nilsson@liu.se}


\newcommand{\LIPSdokumenttyp}{Fördjupningsuppgift \\ i \\ kommunikation och konstruktion}
\newcommand{\LIPSredaktor}{Karl Linderhed, Patrik Nyberg och Erik Nybom}
\newcommand{\LIPSversion}{0.1}
\newcommand{\LIPSdatum}{\dagensdatum}

\newcommand{\LIPSgranskare}{}
\newcommand{\LIPSgranskatdatum}{}
\newcommand{\LIPSgodkannare}{}
\newcommand{\LIPSgodkantdatum}{}

\begin{document}

\LIPStitelsida

%% Argument till \LIPSgruppmedlem: namn, roll i gruppen, telefonnummer, epost
\begin{LIPSprojektidentitet}
  \LIPSgruppmedlem{Karl Linderhed}{Projektledare (PL)}{073-679 59 59}{karli315@student.liu.se}
  \LIPSgruppmedlem{Patrik Nyberg}{Dokumentansvarig (DOK)}{073 -049 59 90}{patny205@student.liu.se}
  \LIPSgruppmedlem{Johan Lind}{}{070-897 58 24}{johli887@student.liu.se}
  \LIPSgruppmedlem{Erik Nybom}{}{070-022 47 85}{eriny778@student.liu.se}
  \LIPSgruppmedlem{Andreas Runfalk}{}{070-564 23 79}{andru411@student.liu.se}
  \LIPSgruppmedlem{Philip Nilsson}{}{073-528 48 86}{phini326@student.liu.se}
  \LIPSgruppmedlem{Lucas Nilsson}{}{073-059 42 94}{lucni395@student.liu.se}
\end{LIPSprojektidentitet}


\renewcommand*\contentsname{Innehåll}
\begin{spacing}{0.5}
\tableofcontents{}
\end{spacing}
\newpage

%% Argument till \LIPSversionsinfo: versionsnummer, datum, ändringar, utfört av, granskat av
%\addcontentsline{toc}{section}{Dokumenthistorik}
\begin{LIPSdokumenthistorik}
  \LIPSversionsinfo{0.1}{2014-03-25}{Första utkast för stilgranskning}{KL}{}
  %\LIPSversionsinfo{1.0}{2014-03-18}{Ändringar efter handledares granskning.}{KL}{-}
\end{LIPSdokumenthistorik}
\newpage

\renewcommand{\thepage}{\arabic{page}}
\setcounter{page}{1}

\section{Inledning}
Denna skrivuppgift tjänar till att fördjupa projektgruppens kunskaper i några områden som är relevanta för projektet. I detta dokument behandlas batterier och servon.

\section{Problemformulering}
Följande frågeställningar ska besvaras i denna uppgift:
\begin{itemize}
\item Hur fungerar litiumjonbatterier?

\item Vilka risker finns vid användning och laddning av litiumjonbatterier?

\item Hur påverkas livstiden hos ett batteri av användning och laddningsmönster?

\item Vilken forskning bedrivs om batterier i nuläget?

\item Hur fungerar ett enkelt servo?

\item Hur kan man få ett servo att röra sig utan kraftiga ryck?
\end{itemize}
\section{Kunskapsbas}
För att samla och inhämta information har gruppen använt sig av vetenskapliga publikationer och litteratur som behandlar ämnena. Detta avsnitt sammanfattar den information som har insamlats.

\subsection{Litiumjonbatterier}
Litium-jon-batterier utnyttjar det faktum att litium är ett av de mest elektronegativa ämnena som finns, och när en elektron lämnar en litiumatom frigörs en stor mängd energi. Detta gör att batterier som använder litiums jonövergång kan få väldigt hög energidensitet och prestanda.

Som i alla batterier används två elektriskt isolerade elektroder, omslutna av en elektrolyt som sammankopplar dem kemiskt. Elektrolyten och materialen i elektroderna möjliggör att kemiska jämviktsreaktioner kan äga rum, och vilken riktning reaktionerna går i avgörs av om det finns en extern potentialskillnad mellan elektroderna eller om de är kortslutna, exempelvis genom någon last, se figur \ref{fig:reaction}. \cite{glaize13}

\nyBild{Reaktionsforlopp.png}{Flöde av ström och joner vid laddning och urladdning}{reaction}{0.8}

\subsection{Servon}

\section{Fördjupning}

\section{Resultat och slutsatser}

\begin{thebibliography}{9}
\bibitem{glaize13} C. Glaize, S. Geniès, \emph{Lithium Batteries and other Electrochemical Storage Systems}\\ Wiley-ISTE, 2013   

\end{thebibliography}
\end{document} 
