
\section{Master transmitt}
\label{bussbilaga}
Pseudokod samt vilka register som ska ändras i vilka steg för att bussen ska få önskad funktion. Denna kommer bestå utav 4 olika program, enligt följande:’

\subsection{Master transmitt}
Programflödet för det program som krävs för master transmitt ska fungera De register som måste modifieras och hur de ska modifieras i de olika punkter finns i följande tabell.

\begin{packed_enumerate}
\item För START ska TWINT samt TWSTA sättas till 1 i TWCR. Processorn kommer då att vänta på att bussen ska bli ledig och sedan skicka START på bussen.
\item Vänta på att TWINT ska gå hög.
\item Läs TWSR, kontrollera innehåll, koderna är som följer:
\begin{packed_enumerate}
\item 0x08 START lyckades.
\item 0x10 REPEATED START lyckades.
\end{packed_enumerate}
\item Skriv SLA+W till TWDR.
\item Skriv TWINT till 1 i TWCR.
\item Vänta på att TWINT ska gå hög.
\item Läs TWSR, kontrollera innehåll, koderna är som följer:
\begin{packed_enumerate}
\item 0x18 SLA+W lyckades, ACK mottagen.
\item 0x20 SLA+W lyckades, NACK mottagen.
\item 0x38 arbitration förlorad.
\end{packed_enumerate}
\item Skicka data till TWDR. Beroende på värde i TWSR ska följa aktion tas:
\begin{packed_enumerate}
\item 0x18: TWINT = 1, TWSTA = TWSTO = 0. Skriver data på bussen.
\item 0x20: TWINT = TWSTA = 1, TWSTO = 0. REPEATED START, tillbaka på punkt 2.
\item 0x38: TWINT = TWSTA = 1, TWSTO = 0. START kommer sändas då bussen är ledig, tillbaka på punkt 2.
\end{packed_enumerate}
\item Vänta på att TWINT till 1 i TWCR.
\item Läs TWSR, kontrollera innehåll, koderna är som följer:
\begin{packed_enumerate}
\item 0x28 Data byte skickad, ACK mottagen.
\item 0x30 Data byte skickad, NACK mottagen.
\end{packed_enumerate}
\item Beroende på kod samt om mer data ska skickas eller inte kan följande aktion tas:
\begin{packed_enumerate}
\item 0x28:
\begin{packed_enumerate}
\item Mer data skickas i samma paket? Om ja, TWINT = 1, TWSTA = TWSTO = 0. Tillbaka på 8 a.
\item TWSTA = TWINT = 1, TWSTO = 0. REPEATED START, om mer data ska skickas, tillbaka på punkt 1.
\item TWINT = TWSTO = 1, TWSTA = 0. STOP skickas, överföring klar.
\end{packed_enumerate}
\item 0x30:
\begin{packed_enumerate}
\item TWINT = TWSTA = 1, TWSTO = 0. REPEATED START skickas, tillbaka på punkt 2.
\item TWINT = TWSTO = 1, TWSTA = 0. STOP skickas, överföring klar.
\end{packed_enumerate}
\end{packed_enumerate}
\end{packed_enumerate}

\subsection{Master receiver mode}
Punkt 1 till 6 samma som för master transmitt mode med skillnaden att SLA+R skrivs till TWDR istället för SLA+W. Sedan följer flödet:

\begin{packed_enumerate}
\setcounter{enumi}{4}
\item Läs TWSR, kontrollera innehåll, koderna är som följande:
\begin{packed_enumerate}
\item 0x38 arbitration förlorad eller NACK mottagen.
\item 0x40 SLA+R skickad, ACK mottagen.
\item 0x48 SLA+R skickad, NACK mottagen.
\end{packed_enumerate}
\item Beroende på värdet i TWSR ska följande aktion tas:
\begin{packed_enumerate}
\item 0x38: TWINT = TWSTA = 1, TWSTO = 0. Vänta till bussen är ledig och skicka START igen, tillbaka på punkt 2.
\item 0x40: Vänta på att TWINT ska bli hög, fortsätt på punkt 9.
\item 0x48: TWINT = TWSTA = 1, TWSTO = 0. REPEATED START kommer att skickas. (Finns risk för oändlig loop om adressen inte finns)
\end{packed_enumerate}
\item Läs TWSR, kontrollera innehåll, koderna är som följer:
\begin{packed_enumerate}
\item 0x50 Data mottagen, ACK skickad.
\item 0x58 Data mottagen, NACK skickad.
\end{packed_enumerate}
\item Läs data i TWDR.
\item Beroende på värdet i TWSR ska följande aktion tas:
\begin{packed_enumerate}
\item 0x50: Ta emot en byte till? Om ja ska TWINT = TWEA = 1, TWSTA = TWSTO = 0. ACK skickas. Om nej ska TWINT = 1, TWEA = TWSTA = TWSTO = 0. NACK skickas.
\item 0x58: Starta en ny överföring? Om ja, sätt TWINT = TWSTA = 1, TWSTO = 0. REPETADE START kommer skickas. Om nej, TWINT = TWSTO = 1, TWSTA = 0.
\end{packed_enumerate}
\end{packed_enumerate}

\subsection{Slave receiver mode}
För att en slav ska kunna ta emot data behövs följande programflöde.

\begin{packed_enumerate}
\item Vänta på TWINT FLAG går hög (hanteras via avbrott)
\item Kontrollera TWSR, möjliga koder är:
\begin{packed_enumerate}
\item 0x60 Egen SLA+W har tagits emot, ACK skickats.
\item 0x68 Förlorat arbitration som master, egen SLA+W/R har mottagits. ACK skickats.
\item 0x70 general call adress mottagen, ACK skickad.
\item 0x78 Arbitration förlorad som master. General call mottaget, ACK returnerad.
\item 0x80 Data har mottagits, ACK returnerad.
\item 0x88 Data har mottagits, NACK returnerad.
\item 0x90 Data har mottagits efter general call, ACK returnerad.
\item 0x98 Data har mottagits efter general call, NACK returnerad.’
\item 0xA0 STOP eller REPETATED START har mottagits medans fortfarande adresserad som slav.
\end{packed_enumerate}
\item Beroende på värde i TWSR ska följande aktion tas:
\begin{packed_enumerate}
\item 0x60: TWINT = TWAE = 1, TWSTO = 0. Data tas emot, ACK returneras.
\item 0x68: Samma som för 0x60.
\item 0x70: Samma som för 0x60.
\item 0x80: Samma som för 0x60.
\item 0x88: Läs från TWDR (kan vara koorupt?) TWINT = TWEA = 1, TWSTO = 0. Ändra inte TWSTA, detta görs i en annan rutin om vi vill skicka på bussen när den blir ledig.
\item 0x90: Läs från TWDR. TWINT = TWEA = 1, TWSTO = 0. Data kommer tas emot, ACK returneras.
\item 0x98: Samma som 0x88.
\item 0xA0: Samma som 0x88.
\end{packed_enumerate}
\end{packed_enumerate}

\subsection{Slave transmitter mode}
För att en slav ska kunna skicka data behövs följande programflöde.

\begin{packed_enumerate}
\item Vänta på att TWINT FLAG går hög. (hanteras via avbrott)
\item Kontrollera TWSR, möjkliga koder är:
\begin{packed_enumerate}
\item 0xA8 Egen SLA+R har tagits emot, ACK skickats.
\item 0xB0 Arbitration förlorad i SLA+R/W som master, egen SLA+R mottagen, ACK skickad.
\item 0xC0 Data i TWDR skickas, NACK mottagen.
\item 0xC8 Sista byte i TWDR skickad, ACK mottagen.
\end{packed_enumerate}
\item Beroende på värde i TWSR ska följande aktion tas:
\begin{packed_enumerate}
\item 0xA8: Ladda data till TWDR. TWINT = TWEA = 1, TWSTO = 0. Data skickas.
\item 0xB0 Samma som 0xA8.
\item 0xB8 Ladda data till TWDR. TWINT = 1, TWSTO = TWEA = 0. Sista data byte skickas.
\item 0xC0 TWINT = TWEA = 1, TWSTO = 0. Går tillbaka till grundläge. Ändra inte TWSTA, detta görs i en annan rutin om vi vill skicka på bussen när det blir ledig.
\item 0xC8 Samma som för 0xC0.
\end{packed_enumerate}
\end{packed_enumerate}


\section{Bluetooth initiering och konfigurering}
\label{Bluetoothinit}

För att inleda kommunikation så behövs följande göras på BT enheten och på PC:

\begin{packed_enumerate}
\item Följa uppkopplingsguide till BT-enheten på följande websida:
https://docs.isy.liu.se/twiki/bin/view/VanHeden/BlueTooth
\item Skicka ‘\$\$\$’ till BT när enheten är Idle (‘stat’-lampa blinkar 1 gång per sek).
\item Konfigurera efter preferens (skicka SF,1 för att gå tillbaka till default).
\item Skicka ‘---’ för att gå ur kommandoläge
\item Nu är det bara att skicka/läsa den data du behagar till/från den port som enheten är kopplad till.
\end{packed_enumerate}

Några bra konfigurationskommandon (ändras endast med välmotiverade och genomtänkta anledningar):

\begin{packed_itemize}
\item SF,1 ;Gör att enheten går tillbaka till defaultinställningar
\item SN  ;Sätter namnet på BTenheten
\item SM  ;Sätter mode på BT
\item SP  ;Sätter pinkod (default är 1234)
\item SU  ;Sätter baud rate (default 115k)
\end{packed_itemize}

För att processorn ska kunna läsa och skriva data till BT-enhet:

\begin{packed_enumerate}
	\item Koppla BT med processorn enligt kopplingschema.
	\item Initiera USART genom att konfigurera alla register enligt:
	\begin{packed_enumerate}
		\item UCSRnB
		\begin{packed_enumerate}
			\item RXENn (bit4) = 0x1, slår på Recieve
			\item TXENn (bit3) = 0x1, slår på Transmitt
			\item RXCIEn (bit7) = 0x1, gör att det sker ett avbrott när processorn fått en recieve.
		\end{packed_enumerate}
		\item USCRnC
		\begin{packed_enumerate}
			\item UMSEL = 0x0 asynkront läge (ingen extern klocka som styr)
			\item UPMn = 0x00, ingen paritet
			\item USBSn = 0x0, vi kör med 1 stopbit
		\end{packed_enumerate}
		\item Baudrate-registrerna konfigureras med formeln, $BaudValue = \frac{F_{CPU}}{16USART_{BAUDRATE} - 1}$ enligt nuvarande konfigurationer sätts:
		\begin{packed_enumerate}
			\item UBRRH = 0x00
			\item UBRRL = 0x09
		\end{packed_enumerate}
	\end{packed_enumerate}
	\item Skicka görs genom att skriva data på UDRn-registret (8 bitar i taget) så görs resten automatiskt. 
Läsning görs genom att läsa från UDRn-registret (8 bitar), detta då en recieve mottagits och inladdning i registret är klart, enligt konfiguration så sker ett avbrott när detta händer.
	\item Se flödesschema för hantering av dataflöde.
\end{packed_enumerate}


