

\section{Programvara – PC}

Gränssnittet ska användas för att en människa ska kunna kommunicera med roboten. Detta innebär att i realtid redovisas alla relevanta styrbeslut som roboten tar, så som fart och riktning. Utöver detta skall gränssnittet även representera världen ur robotens synvinkel, alltså ska det på ett förståeligt sätt uppvisa och formatera de sensordata som sensorenheten samlar in för att ge en bild av lagerrobotens omgivning. För att kunna testa delsystems beteende i grunden finns ett debug-fönster där man kan skicka en förfrågan till ett delsystem för att se vilket svar som returnerades.

\nyBild{Designspec_Disposition_och_arbetsdokument.png}{Användargränssnitt för PC-programvara}{pcgränssnitt}{1}



Programvaran kommer att ha två huvudsakliga uppgifter. För det första så ska den kontinuerligt visa upp relevant data och information från roboten oavsett om roboten opererar i autonomt eller manuellt läge. För det andra så ska programvaran kunna skicka styrbeslut till roboten när den befinner sig i manuellt läge. All kommunikation sköts trådlöst via Bluetooth. Kommunikationsenheten skickar vidare datorns kommandon till rätt enhet. Programvaran ska också periodiskt att skicka ett enkelt paket till roboten som kommer att nollställa en timeout-räknare, för att möjliggöra att roboten kan upptäcka om kontakten förloras.

\subsection{Funktion}

\subsubsection{Styrning}

\subsubsection{Armrörelse}

På instrumentbrädet skall också användaren kunna kontrollera armens rörelse genom knappar eller tangentnedtryckningar. Armens rörelse kommer kunna styras genom rotation av basen och tre leder i själva armen. Gripklon ska man kunna välja att gripa tag och släppa loss med. Dessa styrningar kommer fungera så att roboten rör på sig så länge du håller i knappen för önskad styrning. Det kommer också finnas ett gäng olika fördefinierade lägen för armen som den automatiskt ska kunna röra sig till genom att användaren trycker på önskat läge. Några exempel på fördefinierade lägen är:

\begin{packed_itemize}
\item Startläge, ett infällt läge där roboten kommer vara redo att köra vidare.
\item Avlämning höger/vänster, ett mönster där roboten kommer att lämna av ett föremål på någon sida.
\end{packed_itemize}

\subsubsection{Utslagsgrafer}

\subsubsection{Debugfönster}
Man kommer kunna öppna ett debugfönster där man kan specificera en detaljerad förfrågan bit för bit till ett visst delsystem. När fönstret stängs kommer kommunikationsenheten att skicka denna förfrågan över bussen och få ett svar, som returneras direkt i sin helhet till datorn och visas i loggfönstret.

\subsubsection{Loggfönster}
Loggfönstret kommer fungera som en systemomfattande statusmeddelare. Här kommer ett flöde av beslut och statusrapporter från roboten, bekräftelse av kommando och errormeddelande visas.

\subsubsection{Sensorutslag}


