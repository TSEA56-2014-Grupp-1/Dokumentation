\section{Gemensamma funktioner}
Detta avsnitt beskriver olika funktioner som inbegriper eller används av flera olika moduler.

\subsection{Protokoll för seriell kommunikation mellan robot och PC}
\label{sec:bt-protokoll}
Protokollet som roboten och PC-gränssnittet använder är uppbyggt av olika sorters paket. Varje typ av data som skickas har ett fördefinierat unikt paket-id, som indikerar för en mottagare hur datan ska hanteras. Paket-id:t är alltid den första byten av en överföring. Figur \ref{fig:btpaket} visar hur ett paket är uppbyggt.

\nyBild{Bluetooth-protokoll.png}{Ett paket i den seriella kommunikationen mellan robot och PC.}{btpaket}{0.8}

Antalet

\subsection{USART}
\label{sec:usart}
AVR-processorns inbyggda USART-funktionalitet används av flera delsystem. Därför finns ett gemensamt bibliotek med funktionalitet för att skicka och ta emot bytes över en seriell lina.

Biblioteket använder en byte-buffer för att lagra mottagen data. Till buffern hör två pekare, en läspekare och en skrivpekare, som pekar till olika element i buffern och flyttas upp när man läser eller skriver i buffern. Genom att pekarna endast kan anta värden mellan 0 och 255, samma antal värden som det finns element i buffern, fås funktionalitet som i en ringbuffer eftersom pekarnas värden svämmar över till 0 när de ökas från 255.

När avbrottet som indikerar att en byte har tagits emot kommer skrivs den byte som har tagits emot in i buffern på den position som skrivpekaren pekar på. Därefter ökas pekarens värde ett steg.

\nyBild{USART-buffer.png}{Byte-buffern som USART-biblioteket använder, och vilken data som räknas som tillgänglig och ej ännu läst. En ruta föreställer en byte.}{usartbuffer}{0.6}

För att läsa en byte från buffern anropas funktionen \verb|usart_read_byte|. Den undersöker om det finns någon data tillgänglig som inte redan har lästs (med andra ord, om läs- och skrivpekaren skiljer sig från varandra, se figur \ref{fig:usartbuffer}). I annat fall väntar den tills dess att det finns data, dock som mest under en bestämd timeout-tid. När det finns data läser den in en byte från den position i buffern som läspekaren pekar på till en variabel, och stegar upp läspekaren med ett steg.

För att skicka data används funktionen \verb|usart_write_byte| som undersöker om den inbyggda USART-modulens statusflaggor tillåter att man skickar data, och i sådana fall ger funktionen en byte till USART-modulen som direkt matar ut den på den seriella linan. Går det inte att skicka väntar funktionen till dess att det går innan den skriver datan till USART-modulen.

\subsection{Intern buss}
\label{sec:bus}