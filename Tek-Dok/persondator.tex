\section{Programvara för persondator}
Programvaran till persondatorer som finns till roboten används dels för att ge kommandon till roboten, och dels för att visa utmatning av sensordata och beslut som roboten har tagit. För detaljer om användning, se användarmanualen.

Programvaran är utvecklad i utvecklingsmiljön Qt Creator \cite{qt}, och använder till stor del Qt:s egna standardbibliotek. Den är uppdelad i två klasser, en för att kommunicera med roboten via blåtand och en för det grafiska gränssnittet.

Klassen för blåtandskommunikation använder sig utav QSerialPort, vilket finns i Qt:s standardbibliotek. Ur detta hanterar klassen hur data skickas och tas emot från roboten, och vidarebefordrar denna data till klassen för det grafiska gränssnittet.

Klassen för det grafiska gränssnittet innehåller alla knappar, fält och liknande som finns i det grafiska gränssnittet. Denna använder också till stor del Qt:s egna standardbibliotek. Den hanterar allt som ska hända när en användare interagerar med gränssnittet.

I det grafiska gränssnittet finns också två grafer som ritar ut data från linjesensorn samt sidoskannrar. Dessa använder en klass som heter QCustomPlot, för ytterligare beskrivning se \cite{qcustomplot}.  
