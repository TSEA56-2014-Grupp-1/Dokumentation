\section{Chassienhet}
\emph{Det här avsnittet ska innehålla mera detaljerade blockscheman och beskrivningar av modulen.
Tänk på läsbarheten och växla mellan figurer och text.}

\subsubsection{Överblick}
Chassienheten tar alla övergripande beslut om vad som ska göras, t.ex. är det chassienheten som bestämmer när och på vilken sida armen ska plocka upp och lämna av objekt. Chassienheten sköter styrningen med PD-regleringen med hjälp av den tyngdpunkt som den får av sensorenheten via bussen. Detta för att kunna följa en svart tejpad linje på marken. På chassienheten finns en omkopplare för att kunna välja mellan autonomt och manuellt läge samt två tryckknappar, en tryckknapp för att starta linjeföljningsproceduren samt en för reset. 

\subsubsection{Funktion}

\begin{packed_itemize}
\item Kunna ta beslut och styras både autonomt och genom order från dator via kommunikationsenheten.
\item Styra de två motorerna med PWM-styrning.
\item Använda regleralgoritm för att kunna följa linjen utan att slingra sig fram.
\item Hantera plockstationer genom att stanna och skicka kommandon till arm- och sensorenheten.
\item Kunna vända autonomt.
\item Mäta tid mellan stationer och m.h.a. detta ta beslut om att vända eller inte.
\item Skicka styrbeslut till dator via kommunikationsenheten.
\end{packed_itemize}

\subsection{Kopplingsschema}

%\nyBild{kopplingsschema_chassi.png}{Kopplingsschema för delsystem chassi}{kopplingchassi}{0.8}


\subsection{Komponenter}

\begin{packed_itemize}
\item Färdig plattform med batteri, två motorer som driver varsitt hjul och två “kundvagnshjul”
\item 1 x Atmega1284P, huvudprocessor
\item 2 x tryckknappar för reset samt start/stopp.
\item 1 x omkopplare, för autonomt/manuellt läge.
\item 1 x EXO-3, kristalloscillator (16 MHz)
\end{packed_itemize}

\subsection{Översiktlig beskrivning av programmet}

Programmet på chassit beräknar styrriktning med hjälp av tyngdpunkten den får från sensorenheten . Den hämtar nya värden från sensorenheten med jämna tidsintervall och med dessa uppdaterar motorstyrningen. Enheten räknar också hur många stationer den passerar innan den är tillbaka på stationen den började på. Detta för att roboten ska veta när den är klar. Enheten lyssnar även alltid på styr- och stoppkommandon från PC:n.

Chassit får information om att roboten är på plockstation av sensorenheten i samma paket som den får tyngdpunkten. Vid plockstation stannar roboten och en begäran om att få RFID-tag skickas till sensorenheten. Efter detta skickas en instruktion till armenheten att antingen plocka upp eller lämna av föremål. Alla beslut skickas via kommunikationsenheten till datorn.


\nyBild{huvudprogram.pdf}{Flödesdiagram för huvudprogram till chassienheten.}{chassiflöde}{0.7}

\subsubsection{Följa linje samt motorkontroll}
\label{följalinje}

För att kontrollera hastighet till motorerna används PWM-styrning med en uppdateringshastighet på 1~kHz.

Eftersom systemets klocka kommer gå i 16~MHz måste timern ställas in så att den kan uppnå en period på 1~kHz i uppdateringsfrekvens. Detta uppnås genom att prescalern ($N$) till timern sätts till 8 samt att timern räknar till 1999 innan en ny period ska starta. 

För att kunna göra linjeföljning utan att slingra sig fram implementeras en regleralgoritm. Den reglering som görs är en PD-reglering, enligt formlen:

$$u[t] = K_{p}e[t] + K_{d}\frac{d}{dt}e[t]$$

Om värdet är positivt/negativ ska motor på vänster/höger sida att bromsas medan den andra kör med full fart. För att beräkna motorkontrollen till PWM kommer följande formel användas.

$$P[t] = 1999 - K_{total}u[t]$$

Där alltså $P[t]$ är värdet räknaren till vänster/höger motor ska räkna till. Den andra motorn kommer att gå på full hastighet, alltså är $P[t] = 1999$ för den motorn. Av dessa värden kommer $K_{p}$,$K_{d}$ och $K_{total}$ att kunna ändras för att optimera regleringen.


\subsubsection{Identifiering av plockstationer}

Varje gång roboten stannar vid en plockstation skickar chassit till sensorenheten att läsa av plockstationens RFID-tagg. Värdet hos dessa RFID-taggar sparas i en lista. Detta gör att roboten kan räkna hur många stationer som finns på banan och således vet den när den är klar.


\subsubsection{Styrkommando}

Chassit reagerar alltid på styrkommandon från bussen även om brytaren är satt i autonomt läge. Detta för att kunna korrigera roboten ifall det skulle få för sig att göra något oväntat. Ett styrkommando består av en ändring i riktning och en ändring i gaspådrag. Gaspådrag kan vara positivt och negativt, där positivt för roboten framåt och negativt för roboten bakåt. 


\subsubsection{Autonomt läge och start-/stoppkommando}

När brytaren är i autonomt läge reagerar chassit på startknappen som sitter på chassit. Startknappen påbörjar linjeföljning med hantering av plockstationer beskrivet i \ref{följalinje}.

\subsection{Analys av prestanda och resurser}

Chassienheten använder 2 timers. En för PWM-styrning av motorerna och en som ger avbrott för att begära linjesensordata med jämna intervall. En av ATmega1284:ans 8~bits-timers används för att begära sensordata, och en 16~bits-timer används till PWM.
