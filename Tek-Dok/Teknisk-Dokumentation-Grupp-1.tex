\documentclass[a4paper,12pt]{article}
\usepackage{graphicx}

\usepackage{epstopdf}
\usepackage{gensymb}
\usepackage{float}
\usepackage{mathtools}
\usepackage{setspace}
\usepackage{tabularx}
\title{Teknisk Dokumentation}
\input{LIPS.tex}

\usepackage{sectsty}
\allsectionsfont{\sffamily}

\frenchspacing

\renewcommand{\thepage}{\roman{page}}

\newcommand{\LIPSartaltermin}{VT14}
\newcommand{\LIPSkursnamn}{TSEA56 Elektronik kandidatprojekt}

\newcommand{\LIPSprojekttitel}{Lagerrobot}

\newcommand{\LIPSprojektgrupp}{Grupp 1}
\newcommand{\LIPSgruppepost}{tsea56-2014-grupp-1@googlegroups.com}
\newcommand{\LIPSdokumentansvarig}{LIPS Teknisk Dokumentation}

\newcommand{\LIPSkund}{ISY, Linköpings universitet, 581\,83 Linköping}
\newcommand{\LIPSkundkontakt}{Tomas Svensson, 013-28 13 68, tomass@isy.liu.se}
\newcommand{\LIPSkursansvarig}{Tomas Svensson, 013-28 13 68, 3B:528, tomass@isy.liu.se}
\newcommand{\LIPShandledare}{Anders Nilsson, 3B:512, 013-28 26 35, anders.p.nilsson@liu.se}


\newcommand{\LIPSdokumenttyp}{Teknisk Dokumentation}
\newcommand{\LIPSredaktor}{Karl Linderhed}
\newcommand{\LIPSversion}{1.0}
\newcommand{\LIPSdatum}{\dagensdatum}

\newcommand{\LIPSgranskare}{}
\newcommand{\LIPSgranskatdatum}{}
\newcommand{\LIPSgodkannare}{}
\newcommand{\LIPSgodkantdatum}{}

\setlength{\parskip}{\baselineskip}%
\setlength{\parindent}{0pt}%

\begin{document}

\LIPStitelsida

%% Argument till \LIPSgruppmedlem: namn, roll i gruppen, telefonnummer, epost
\begin{LIPSprojektidentitet}
  \LIPSgruppmedlem{Karl Linderhed}{Projektledare (PL)}{073-679 59 59}{karli315@student.liu.se}
  \LIPSgruppmedlem{Patrik Nyberg}{Dokumentansvarig (DOK)}{073 -049 59 90}{patny205@student.liu.se}
  \LIPSgruppmedlem{Johan Lind}{}{070-897 58 24}{johli887@student.liu.se}
  \LIPSgruppmedlem{Erik Nybom}{}{070-022 47 85}{eriny778@student.liu.se}
  \LIPSgruppmedlem{Andreas Runfalk}{}{070-564 23 79}{andru411@student.liu.se}
  \LIPSgruppmedlem{Philip Nilsson}{}{073-528 48 86}{phini326@student.liu.se}
  \LIPSgruppmedlem{Lucas Nilsson}{}{073-059 42 94}{lucni395@student.liu.se}
\end{LIPSprojektidentitet}


\renewcommand*\contentsname{Innehåll}
\begin{spacing}{0.5}
\tableofcontents{}
\end{spacing}
\newpage

%% Argument till \LIPSversionsinfo: versionsnummer, datum, ändringar, utfört av, granskat av
%\addcontentsline{toc}{section}{Dokumenthistorik}
\begin{LIPSdokumenthistorik}
  \LIPSversionsinfo{1.0}{}{}{}{}
\end{LIPSdokumenthistorik}
\newpage

\renewcommand{\thepage}{\arabic{page}}
\setcounter{page}{1}

\section{Inledning}
Sju studenter vid Linköpings Universitet valde i deras kandidatprojektkurs TSEA56 att konstruera en automatiserad lagerrobot utifrån beställarens krav på design och funktion.

Projektet gjordes med avsikten att ge studenterna bildning och erfarenhet att utifrån de kunskaper erhållna från tidigare kurser kunna genomföra ett omfattande uppdrag med samma metodik som förekommer i arbetslivet.


\section{Produktbeskrivning}
Produkten är en robot ihopbyggd av ett chassi av plexiglas varpå fyra hjul, sensorer, en arm och kretskort är fastmonterad, utöver detta finns en programvara för övervakning och styrning av roboten.

Roboten har som avsikt att användas i ett lager där den genom beordning manuellt eller autonomt hämtar och lämtar lagervaror. Lagerarbetet görs genom att följa svartmarkerad linje på marken där stationer är markerade med ett svart sträck rakt ut åt det håll stationen är tillsammans med en RFID-tag för att roboten att kunna identifiera stationen. Utefter preferens kan den styras manuellt med programvaran eller köra och hämta/lämna föremål automatiskt.

\section{Teori}
\emph{Beskrivning av regleralgoritmer m.m.}

\section{Systemöversikt}
Roboten är uppbyggd av fyra olika delsystem (chassi, sensor, arm och kommunikation) som vart och ett har sin specifika uppgift att sköta. Förutom robotens fyra delsystem har också en programvara till en dator utvecklats för övervakning och styrning. Roboten kan både operera i ett fjärrstyrt och i ett autonomt läge, här kan man välja om man endast vill ha automatisk linjeföljning, endast automatiskt föremålsupplockning eller om man vill ha all funktionallitet automatisk.

\subsection{Delsystemsöversikt}
Sensorenheten övervakar kontinuerligt de olika sensorerna roboten har och omvandlar den råa sensordatan till systemanvändbara enheter för att sedan kunna skicka dessa vidare till andra delsystem vid förfrågan.

Kommunikationsenheten fungerar som en förbindelselänk mellan roboten och omvärlden. Den kan bli tillfrågad och skicka bekräftelse om de beslut som roboten gör samt status över de olika delsystemen och omgivningsobservationer. Vidare är kommunikationsenheten utrustad med en display vilken används för att visa upp en del av den sensordata som samlas in.

Chassienheten fungerar som det handlingsbeslutande organet på roboten. Den vet vid autonomt läge hur roboten ska agera vid olika situationer. Den styr också hjulen på roboten, det vill säga att den genom sensordata och reglering beräknar hastighet på hjulen för att kunna svänga och hålla kursen längst banan.

Armenheten kan vid en avhämtningsstation, utifrån identifiering av var ett objekt befinner sig, kunna plocka upp detta med robotarmen som är monterad på chassit. Den har kontroll över samtliga servon och vet hur dessa ska röras vid kommando. Med hjälp av inverterad kinematik kan den, vid automatiskt läge, beräkna hur vardera led skall röra sig för nå objektet och för att plocka upp det.

Den datorbaserade programvaran består av ett instrumentbräde där användaren kan se robotens vilka beslut som tas av roboten samt vilka sensordata de olika sensorerna tar in. Vidare kan roboten styras utifrån det grafiska gränssnittet, med avseende på både robotarm och chassi.

\subsection{Sensorer och sensorplacering}

En linjesensor sitter längst fram på roboten, nära marken. En RFID-läsare sitter parallellt med marken under roboten, då den endast har 7~cm läsavstånd till taggarna som ligger på marken. På vardera sida av roboten är en avståndssensor placerad på ett servo, för att skanna efter objekt att plocka upp.


\section{Gemensamma funktioner}
Detta avsnitt beskriver olika funktioner som inbegriper eller används av flera olika moduler.

\subsection{Protokoll för seriell kommunikation mellan robot och PC}
\label{sec:bt-protokoll}
Protokollet som roboten och PC-gränssnittet använder är uppbyggt av olika typer av paket. Varje typ av data som skickas har ett fördefinierat unikt paket-ID, som indikerar för en mottagare hur datan ska hanteras. Paket-ID:t är alltid den första byten av en överföring. Figur \ref{fig:btpaket} visar hur ett paket är uppbyggt.

\nyBild{Bluetooth-protokoll.png}{Ett paket i den seriella kommunikationen mellan robot och PC, varje ruta är en byte lång.}{btpaket}{0.8}

Paketlängden som skickas med är antalet parametrar + 1, eftersom även kontrollsumman ingår i paketet. Paketlängden indikerar alltså hur många byte som finns kvar i paketet efter paketlängden.

Kontrollsumman bildas genom att summera värdena av alla bytes i paketet, exklusive kontrollsumman själv, trunkera summan till 8 bitar, och invertera värdet bitvis.

\subsection{USART}
\label{sec:usart}
AVR-processorns inbyggda USART-funktionalitet används av flera delsystem. Därför finns ett gemensamt bibliotek med funktionalitet för att skicka och ta emot data över en seriell lina.

Biblioteket använder en databuffer för att lagra mottagen data. Till buffern hör två pekare, en läspekare och en skrivpekare, som pekar till olika element i buffern och flyttas upp när man läser eller skriver i buffern. Genom att pekarna endast kan anta värden mellan 0 och 255, samma antal värden som det finns element i buffern, fås funktionalitet som i en ringbuffer eftersom pekarnas värden svämmar över till 0 när de ökas från 255.

När avbrottet, som indikerar att en byte har tagits emot, sker kommer den byte som tagits emot skrivas in i buffern på den position som skrivpekaren pekar på. Därefter ökas pekarens värde ett steg.

\nyBild{USART-buffer.png}{Byte-buffern som USART-biblioteket använder, och vilken data som räknas som tillgänglig och ej ännu läst. En ruta föreställer en byte.}{usartbuffer}{0.6}

För att läsa en byte från buffern anropas funktionen \verb|usart_read_byte|. Den undersöker om det finns någon data tillgänglig som inte redan har lästs (med andra ord, om läs- och skrivpekaren skiljer sig från varandra, se figur \ref{fig:usartbuffer}). I annat fall väntar den tills dess att det finns data, dock som mest under en bestämd timeout-tid. När det finns data läser den in en byte från den position i buffern som läspekaren pekar på till en variabel, och stegar upp läspekaren med ett steg.

För att skicka data används funktionen \verb|usart_write_byte| som undersöker om den inbyggda USART-modulens statusflaggor tillåter att man skickar data, och i sådana fall ger funktionen en byte till USART-modulen som direkt matar ut den på den seriella linan. Går det inte att skicka väntar funktionen till dess att det går innan den skriver datan till USART-modulen.

\subsection{Intern buss}
\label{sec:bus}

För att delsystemen internt ska kunna kommunicera med varandra finns en intern buss av typen multimaster, se avsnitt \ref{sec:bus-implementation}, I$^2$C\footnote{Inter-Integrated Circuit, en synkron seriell multimasterbuss.} implementerad.

\subsubsection{Protokoll}
\label{sec:bus-protokoll}
Alla transaktioner består av en adress med en skriv- eller läsbit, följt av två byte med data. Då enheten som initierat kontakten (master) vill skriva ut på bussen består de första 5 bitarna av ett ID. Detta ID bestämmer vilken funktion hos mottagaren, som alltså agerar slav, som ska hantera den data som överförs i de resterande 11 bitarna.

\nyBild{Buss-protokoll.pdf}{En lyckad skrivning från master till slav. Om NACK (No Acknowledgement, innebär att mottagaren ej mottagit data) skulle ha returnerats istället för ACK (Acknowledgement innebär att mottagaren tog emot data) kommer överföringen att avbrytas.}{busstransaktion}{1}

I fallet att master vill göra en förfrågan på bussen gör den först en skrivning till slaven. Detta gör då att slaven kör funktionen på det ID den fick med de resterande 11 bitar som inargument. Denna funktion returnerar ett 16 bitars värde vilket kommer vara det värde som skickas ut då mastern sedan begär en läsning från slaven. Mellan dessa två transaktioner är det viktigt att mastern inte släpper kontroll över bussen, eftersom slaven då skulle kunna returnera fel värde om en annan enhet skrivit till den efter den första transaktionen.

\subsubsection{Implementation}
\label{sec:bus-implementation}
Bussen är implementerad så att alla enheter kan initiera en transaktion och därmed kan alla enheter vara master på bussen. Detta gör att två eller fler enheter kan börja kommunicera på bussen samtidigt. Då detta sker kommer den enhet som skickar en låg bit först att vinna bussen och fortsätta med sin transaktion. De andra enheterna kommer då upphöra med sina transaktioner och efter att ett stop skickats på bussen kommer de att försöka igen. Detta kommer enheterna att göra tills dessa att de lyckats med en transaktion.

\begin{table}[H]
\centering
\begin{tabularx}{0.5\textwidth}{|X|X|}
\hline
\textbf{Enhet} & \textbf{Adress} \\ \hline
Sensor & 4 \\ \hline
Kommunikation & 5 \\ \hline
Chassi & 1 \\ \hline
Arm & 6 \\ \hline
\end{tabularx}
\caption{Adresser till de olika delsystem på den interna bussen.}
\label{tab:adress-buss}
\end{table}

I tabell \ref{tab:adress-buss} finns adresserna till de olika delsystemen. Adresserna har valts så att en prioritering av transaktioner till de olika delsystemen finns då flera använder bussen samtidigt. Det betyder i detta fall att transaktioner till chassit alltid har högst prioritet.

När en enhet gör en transaktion på bussen och enheten som tar emot data returnerar NACK efter någon byte i överförningen kommer transaktionen att avbrytas. Detta innebär även att en ny transaktion inte kommer att startas per automatik.

För att bestämma vilken funktion som ska hantera data på vilket ID finns funktionerna \texttt{bus\_register\_receive}, som registrerar en funktion för mottagen data, samt \texttt{bus\_register\_response} som registrerar en funktion då data ska returneras över bussen.

De ID:n som finns implementerade på de olika delsystemen och vad de gör finns i bilaga \ref{callbacks}.

Funktionen för att skicka data på bussen är \verb|bus_transmit|, och för att göra en förfrågan finns \verb|bus_request|. Båda dessa returnerar 0 då transaktionen lyckades. Detta innebär att om användaren vill vara säker på att en transaktion ska gå fram på bussen måste dessa funktioner köras tills dess att de returnerar 0.

Hastigheten på bussen har valts till så låg som möjligt, detta för att få maximal stabilitet. Eftersom hastigheten sätts genom en division av klockhastigheten till processorn kommer bussen att gå olika fort beroende av vilken enhet som är master. Detta innebär att hastigheten varierar mellan cirka 70 och 90 kHz.

\subsection{Utmatning på LCD-skärm}
\label{sec:lcd_interface}

Alla enheter kan mata ut information på kommunikationsenhetens LCD-skärm. Detta görs genom biblioteket \verb|lcd_interface| och funktionen \verb|display|. Funktionen kan ta godtyckligt antal parametrar, dock som minst två där den första anger vilken av LCD-skärmens två rader man vill använda och den andra är en sträng med texten som ska skrivas ut. Övriga parametrar är variabler som man vill mata ut, dessa refereras till i textsträngen med samma syntax som är standard i funktionen \verb|printf| i C.

LCD-skärmen kommer att rotera mellan de fyra olika enheternas meddelanden med ca två sekunders intervall. Varje ''sida'' som visas på skärmen identifieras med en bokstav för vilken enhet som visas för tillfället: ''C'' för chassienheten, ''S'' för sensorenheten, ''K'' för kommunikationsenheten och ''A'' för armenheten.

\section{Kommunikationsenhet}
Kommunikationsenheten är robotens gränssnitt för interaktion med omvärlden och har två uppgifter. Den ska
\begin{itemize}
\item agera som ett gränssnitt mot LCD-skärmen och möjliggöra att alla delsystem kan skriva ut information på den.
\item hantera den seriella kommunikationen över blåtand och möjliggöra att roboten kan skicka och ta emot information till och från PC-gränssnittet.
\end{itemize}

Figur \ref{fig:kommblock} ger en övergripande bild över kommunikationsenhetens beståndsdelar.

\nyBild{Block-komm.png}{Blockschema över kommunikationsenheten.}{kommblock}{0.9}

\subsection{LCD-gränssnitt}
LCD-skärmen som är monterad på kommunikationsenheten är av modellen JM162A. Den styrs genom en parallell databuss och ett antal styrsignaler från kommunikationsenhetens processor till en intern processor i LCD-enheten. Skärmen har två rader med 16 tecken vardera. \cite{lcd}

Information skickas till LCD-enheten antingen som instruktioner eller data. Styrsignalernas koppling visas i figur \ref{fig:kommblock}. \verb|RS| styr om databussens värde tolkas som instruktion eller data, \verb|R/W| styr ifall information ska skrivas eller läsas, och styrsignalen \verb|E| aktiverar överföring av information. Genom att låta \verb|E| gå från hög till låg läser LCD-enheten in värdet som finns på databussen, antingen som en instruktion som ska utföras (om \verb|RS| är låg) eller som data som ska lagras i LCD-enhetens minne (om \verb|RS| är hög). Figur \ref{fig:lcdinit} visar vilka kommandon som skickas för att initiera LCD-enheten.

\nyBild{lcd-init.png}{Flödesschema över initialisering av LCD-enhet. Kommandon som skickas är kursiverade och har värdena på DB (i binär notation), RS och R/W som parametrar.}{lcdinit}{0.4}

Efter denna initiering kan data skrivas ut på skärmen genom att sätta \verb|RS| till 1 och skriva ut ASCII-koden\footnote{American Standard Code for Information Interchange, ett sätt att koda grundläggande alfanumeriska tecken och andra symboler.} för en symbol på databussen. Genom att först skicka en instruktion för att ställa in vilken adress i dataminnet som ska skrivas till härnäst kan man välja var en symbol ska skrivas ut, en viss adress i dataminnet svarar mot en position på skärmen.

Alla delsystem kan skriva ut information på skärmen genom ett gemensamt bibliotek, \textit{lcd\_interface}, som beskrivs närmare i avsnitt \ref{sec:lcd_interface}.


\subsection{Seriell kommunikation över blåtand}
Anslutningen till datorn sker med blåtandsmodemet \emph{BlueSMiRF Gold} som är monterat på roboten \cite{bluetooth}. På PC-sidan skapas vid parkoppling med modemet en virtuell serieport, som emulerar en fysisk COM-port eller motsvarande. Kommunikationen sker sedan över denna port som om den vore en vanlig RS-232-port\footnote{RS-232 är en typ av seriell kommunikation vid överföring av data \cite{rs232}.}. 

På robotsidan används AVR-processorns inbyggda modul för USART, och det gemensamma biblioteket för USART som används av flera delsystem -- se avsnitt \ref{sec:usart}. Följande parametrar ställs in i det protokoll som roboten använder:
\begin{itemize}
\item Datahastigheten är 115 200 bps.
\item Data skickas som 8-bitars värden utan någon paritetsbit.
\item Ingen flödeshantering eller handskakning används.
\item Varje värde om 8 bitar avslutas med en stoppbit.
\end{itemize}

Hur olika typer av information överförs mellan robot och PC beskrivs närmare i avsnitt \ref{sec:bt-protokoll}.

\section{Sensorenhet}
\emph{Det här avsnittet ska innehålla mera detaljerade blockscheman och beskrivningar av modulen.
Tänk på läsbarheten och växla mellan figurer och text.}

\subsection{Funktion}
Sensorenhetens uppgift är att samla in rådata från de olika sensorerna och formatera denna till användbar information. På begäran av andra delsystem skickar sensorenheten ut data via bussen. Sensorenheten är utrustad
med en linjesensor vars uppgift är att ge information om robotens befinnande i förhållande till tejplinjen. Informationen används som styrdata och skickas över bussen i form av en tyngdpunkt till chassimodulen för styrreglering.

Vidare är roboten bestyckad med två stycken sidoskannrar på vardera sida om den. Dessa används för att lokalisera objektet roboten ska plocka upp. Om ett föremål hittas räknar sensorenheten ut en koordinat för föremålet och skickar koordinaten över bussen till armenheten.

\subsection{Kopplingsschema}
\nyBild{kopplingsschema_sensor.png}{Kopplingsschema för sensorenheten}{senskoppling.}{0.6}

\subsection{Komponenter}
\begin{packed_itemize}
\item 1 x ATmega1284P, huvudprocessor
\item 1 x refelxsensormodul
\item 2 x 16-kanals multiplexrar
\item 1 x RFID-läsarer (Parallax Serial)
\item 2 x avståndssensorer (GP2D120)
\item 2 x resistorer (18~k\ohm), för lågpassfilter
\item 2 x kondensatorer (100~nF), för lågpassfilter
\end{packed_itemize}

\subsection{Sensorer}
Sensorenheten använder tre olika sensortyper: avståndssensor, linjesensor och RFID-läsare.
Dessa beskrivs kortfattat nedan. En djupare beskrivning av hur komponenterna fungerar finns i bilaga "sensorfördjupning"

\subsubsection{Avståndssensorer}
Avståndssensorerna används till robotens sidoskannrar. De mäter avstånd genom att skicka ut infrarött ljus som sedan reflekteras tillbaka i en PSD, Position Sensitive Detector. Avståndssensorerna som används fungerar för avstånd mellan \mbox{$4-30$ cm}. Eftersom de först mäter ett digitalt värde och sedan skickar ut en analog spänning uppstår brus på utsignalen. Detta brus är relativt högfrekvent varför det filtreras bort med hjälp av ett lågpassfilter innan signalen når processorn. Se kopplingsschema, figur \ref{senskoppling}.


En avläsning från en avståndssensor görs genom att omvandla sensorns analoga utspänning till ett digitalt värde. När A/D-omvandlingen är klar får sensorenheten ett avbrott där den jämför den A/D-omvanldade spänningen med en tabell över kända avstånd och spänningar för att översätta den till ett avstånd i millimeter. Värden som inte finns i tabellen beräknas med linjärinterpolering. 

Avståndssensorerna är någorlunda precisa i sina mätningar inom intervallet de är specificerade för. Men när en koordinat för ett föremål ska kalkyleras baserat på avståndssensorns uppmätta värden blir små avvikelser ännu mer kritiskt. Olika föremål och underlag kan få avståndssensorn att ge ut olika spänningar för samma avstånd. IR-ljuset som sänds ut har också en viss spridning men genom att montera avståndssensor i vertikalled minskar spridningspåverkan. Se figur \ref{fig:sidoskanner-montering}

\subsubsection{Linjesensor}
Reflexsensormodulen består av en uppsättning IR-dioder och fototransistorer för att mäta ljusheten hos underlaget. Eftersom reflexsensormodulen består av 11~stycken separata IR-dioder och fototransistorer används en multiplexer och en demultiplexer för att styra sensorn. Demultiplexern driver IR-dioderna och multiplexern används för att ta in insignalerna från fototransistorerna. Linjesensorn kommer att vara kopplad i enlighet med kopplingsschemat i figur \ref{fig:senskoppling}.

Reflexsensorn ger ut en analog spänning som är omvänt proportionell mot underlagets reflekterande ljusstyrka. En inläsning från linjesensorn sker genom ett funktionsanrop (via avbrott från A/D-omvandlaren) till linjesensorinläsningsfunktionen som finns på sensorenheten. När denna funktion anropas uppdateras en uppsättning variabler som är lagrade i en vektor. Vektorn är 11 element lång, där varje element representerar en reflexsensor.

När uppdateringen av linjesensorn startas för första gången är muxarna som styr linjesensorn inställda så att reflexsensorn längst till vänster väljs. Efter detta startas A/D-omvandling på kanal~0. När A/D-omvandlingen är färdig kommer ett avbrott (ADIF) att genereras vilket triggar en avbrottsrutin.

I denna avbrottsrutin skrivs det A/D-omvandlade värdet in på korrekt plats i linjesensorns datavektor. Denna plats motsvarar den valda linjesensorns position på sensormatrisen. Därefter upprepas samma process för nästkommande reflexsensor. Figur \ref{fig:senslinjeflöde} illustrerar processen för att uppdatera linjesensorns data.

\subsubsection{RFID-läsare}
Den RFID-läsare som kommer att användas är en “Parallax Serial”, denna kräver två pinnar på processorn enligt kopplingsschema i figur \ref{fig:senskoppling}.

\subsection{Sidoskanner}
Sidoskannerns uppgift är att hitta det objekt som ska plockas upp när roboten har stannat vid en plockstation. Sidoskannern består av två avståndssensorer, på höger-respektive vänster sida om roboten, monterade på varsitt servo. Servona styrs med snabb pulsbreddsmodulering. Servot får en puls var 20:e~millisekund och pulsbredden avgör vilket läge servot antar. En pulsbredd på ungefär 0.5~ms motsvarar servots ursprungsläge och en pulsbredd på ungefär 2.5~ms motsvarar max vinkelutslag. Detta varierar dock något från servo till servo.

För att få ett servot att svepa över robotens ena sida ökas pulsbredden inkrementellt med en stegkonstant motsvarande ett servoutslag på en grad. För varje iteration, dvs för varje vinkel servot står i, kommer avståndssensorn lägga in 20 stycken A/D-omvandlade avståndsmätningar i en array för att sedan ta medianvärdet av mätningarna. Slutligen omvandlas värdet till ett avstånd i millimeter med interpolationsfunktionen. Om avståndet är inom en zon, baserat på armens räckvidd, innebär det att att inget föremål detekterats och servots vinkel stegas upp ytterligare en grad. 

När avståndssensorn däremot påträffar ett föremål inom zonen sparas både vinkeln som servot står i och avståndet till föremålet undan innan sidoskannern stegar upp igen. För varje vinkel som avståndssensorerna fortfarande träffar föremålet sparas avståndet undan och slutligen även den sista vinkeln då den fortfarande träffade ett föremål.

En medelvinkel av den första och sista vinkeln räknas ut och ett medianvärde för avståndet till föremålet. Medelvinkeln $\alpha$ och medianavståndet $L$ används sedan för att räkna ut en vinkel $\beta$ och ett avstånd, $R$, för armen till föremålet innan dessa skickas tillbaka till armenheten. Se figur \ref{fig:sidoskanner}
\nyBild{sidoskanner.png}{Vänster sidoskanner.}{sidoskanner}{0.7}

\subsection{Översiktlig beskrivning av programmet}

\subsubsection{Inläsning från avståndssensorer}








\section{Chassienhet}
\emph{Det här avsnittet ska innehålla mera detaljerade blockscheman och beskrivningar av modulen.
Tänk på läsbarheten och växla mellan figurer och text.}

\subsection{Överblick}
Chassienheten tar alla övergripande beslut om vad som ska göras, t.ex. är det chassienheten som bestämmer när och på vilken sida armen ska plocka upp och lämna av objekt. Chassienheten sköter styrningen med PD-regleringen med hjälp av den tyngdpunkt som den får av sensorenheten via bussen. Detta för att kunna följa en svart tejpad linje på marken. På chassienheten finns en omkopplare för att kunna välja mellan autonomt och manuellt läge samt en tryckknapp för att starta linjeföljningsproceduren. 

\subsection{Funktion}

\begin{packed_itemize}
\item Kunna ta beslut och styras både autonomt och genom order från dator via kommunikationsenheten.
\item Styra de två motorerna med PWM-styrning.
\item Använda regleralgoritm för att kunna följa linjen utan att slingra sig fram.
\item Hantera plockstationer genom att stanna och skicka kommandon till arm- och sensorenheten.
\item Kunna vända autonomt.
\item Mäta tid mellan stationer och m.h.a. detta ta beslut om att vända eller inte.
\item Skicka styrbeslut till dator via kommunikationsenheten.
\end{packed_itemize}
\section{Armenhet}

Robotens arm är av modell PhantomX Reactor från Trossen Robotics, som är en servostyrd arm med 4 rotationsleder och en griphand varpå det sitter totalt 7~st AX-12A-servon. Armen kontrolleras av en microprocessor, ATmega 1284, genom att parallellkoppla servona till en seriell UART-port. Kommunikationen till servona sker via half duplex via en tri state buffer. Armen har en maximal räckvidd på 38~cm och en bas med  300~grader rotationsfrihet.

\subsection{Funktion}

Armen kan styras manuellt via PC-programvaran och kan köras i autonomt läge. Enhetens huvudprogram är avvaktande tills dess att kommando skickas till enheten vilket kan antingen vara ett manuellt styrningskommando eller en order från chassienheten att ta hand om upplockning av objekt.

\subsubsection{Manuellt läge} 

Vid manuellt läge får armen ingen upplockningsorder från chassienheten och väntar därför endast efter ett styrkommando från PC. Från PC får den ett kommando att röra sig i en viss riktning i koordinatsystemet visat i figur \ref{fig:arminverskinematik}.

Användaren kan välja att röra armen i djupled (Y-axeln), i höjd (Z-axeln), rotera runt Z-axeln eller öppna och stänga klon. Vid ett styrkommando rör sig armen i den angivna riktningen tills dess att ett kommando om att stanna rörelse i den riktingen ges från PC eller om den nått max-läget. Rörelse görs genom att kontinuerligt beräkna hur vinkeln av vardera led på armen skall ändras för att röra sig ett litet steg i den beordrade riktningen, detta görs genom beräkningar i inverterad kinematik (se \ref{inverskinematik}), sedan gör lederna en vinkelförändring för att sedan repetera enligt ovan.

Armen kan också via PC automatiskt röra sig till ett förbestämt läge, ett startläge, där rörelsen även här beräknas med inverterad kinematik till den position som startläget har. Först rör sig armen rakt upp från det nuvarande läget för att sedan röra sig till den position som startläget har, detta för att eliminera risken att stöta i chassit och virkorten.

\subsubsection{Autonomt läge}

Vid autonomt läge avvaktar armenheten tills dess att chassit har identifierat att vi befinner oss på en plockstation som är aktuell för upplockning eller avlämning. Vid station får armenheten en order, plocka upp eller lämna av objekt till höger eller till vänster. När ordern inkommer och armen inte redan håller i ett föremål tillkallar den sensorenheten att med, beroende på vilken sida stationen befinner sig på, en av sidoscannrarna söka igenom upplockningszonen. Håller armen redan i ett föremål rör den sig till samma koordinat som för upplockningen av samma objekt, släpper föremålet och rör sig sedan tillbaka till startläget för att sist skicka bekräftelse till chassit med status om avlämning lyckats.

Sensorenheten kommer, när den är klar med sidoscannrarna, skicka tillbaka ett kommando till armenheten med en vinkel och ett avstånd till föremålet, om den hittat något, annars meddelar den att zonen är tom. Med vinkeln och avståndet beräknar armenheten först en koordinat till objektet och sedan, med inverterad kinematik, hur armens olika leder behöver röra på sig för att nå koordinaten. Armen rör sig sedan mot objektet och väntar på att servona ska nå slutdestinationen, genom att fråga servona om de rör på sig, för att sedan gripa tag och röra sig tillbaka till startläget. Armenheten sparar sedan koordinaten där den plockat upp föremål, vilket används som avlämningsposition, och skickar bekräftelse till chassienheten om att upplockning av föremål lyckats. 

\subsubsection{Inverterad kinematik}
\label{inverskinematik}

BEHÖVS SKRIVAS OM, KOPIERAT FRÅN DEC-SPEC

Positioner anges i ett koordinatsystem relativt armens bas. X anger avstånd i sidled, Y anger avstånd i djupled och Z anger avstånd i höjdled. Armens gripklo hålls horisontell mot underlaget. Då armens rotation i XY-planet endast styrs av armbasens rotation kan det inverterade kinematikproblemet reduceras till ett tvådimensionellt problem. Eftersom gripleden ska hållas horisontell kan även denna led tas ur ekvationen, då den anpassas efter de övriga två ledernas position.

%\nyBild{Figur_arm_--_inveterad_kinematik.png}{Figur över koordinatsystem och vinklar som %definierar armens läge.}{arminverskinematik}{1}

I figur \ref{fig:arminverskinematik} ses att origo ligger över marknivå. $\gamma$ kommer vara inställd i den riktning som armen ska röra sig i för att nå föremålet. $\beta$ kommer se till att hålla $L_{3}$ horisontell mot marknivån.

$$
\begin{pmatrix}
	x \\
	y
\end{pmatrix}
 = 
\begin{pmatrix}
	L_{1}cos(\alpha_{1}) + L_{2}cos(\alpha_{1} + \alpha_{2} - \pi) \\
	L_{1}sin(\alpha_{1}) + L_{2}sin(\alpha_{1} + \alpha_{2} - \pi)
\end{pmatrix}$$

\subsection{Kopplingsschema}
%\nyBild{kopplingsschema-arm}{Kopplingsschema över delsystem arm}{kopparm}{1}


\subsection{Komponenter}

\begin{packed_itemize}
\item 1 x PhantomX Reactor arm med 7st AX-12A-servon
\item 1 x 3-state buffer (74LS241)
\item 1 x ATMega1284P, huvudprocessor
\item 1 x EXO-3, kristalloscillator (16~MHz)
\item 1 x Knapp med Resistor, för pullup av reset
\item 1 x Kondensator, för fördröjning av resetpullup vid start
\end{packed_itemize}


\subsection{Översiktlig beskrivning av programmet}

Programmet kommunicerar med servona över UART via funktioner som finns i ett delat bibliotek för denna typ av kommunikation. Servona styrs genom att skriva på deras minnen, där en skrivning på särskilld plats i minnet leder till en särskilld instruktion till servot. För att skriva eller läsa från servona skapas i programmet först ett paket, med utseende beroende på instruktion, med samtliga bitar som ska skickas för att sedan seriellt skicka dem via UART. Vid en överföring, borstett från när instruktion skickas till alla servon, väntar programmet på att få svar från servot för att kunna diagnostisera om instruktionen kunnat genomföras eller för att tyda en avläsning av servominne. Svar omhändertas genom väntan på mottagningsavbrott på armenheten därpå inkommande data lagras i en mottagningsbuffer som läses av så fort den inte är tom.

Eftersom armenheten är avvaktande utför huvudprogrammet endast instruktioner när statusflaggor blivit satta genom funktionskallelser från andra enheter på bussen. 

Med hjälp av inverterad kinematik kan programmet beräkna vilka vinklar armens servon ska anta för att armens gripklo ska anta en given position.

Nedanstående figur visar vilka bibliotek som programmet består av. Grönmarkerade lådor innebär att de kan delas med andra delar av roboten. Lådor som ligger ovanför en annan låda bygger på den lådans funktioner. Exempelvis är servostyrning och servoläsning beroende av funktioner för att kommunicera över UART.

%\nyBild{Programstruktur_arm.png}{Övergripande bild över armenhetens programbibliotek.}{armprogrambibliotek}{0.8}


\section{Slutsatser}
\emph{Vilka förbättringar skulle kunna göras?}
Att låta roboten använda sig utav någon form av trippmätare, alltså en sensor som kan mäta robotens tillryggalagda sträcka, hade kunnat göra så att roboten klarar av banan bättre. Om roboten har information om hur lång sträcka som färdats skulle ett beslut om kortaste vägen till varje avlämningsstation kunna tas. Detta skulle reducera den sträcka roboten måste transportera sig, och således även den tid det tar att utföra uppdraget.
\newpage
\section*{Referenser}


\newpage
\appendix


\section{Kravspecifikation}
\label{sec:kravspec}

\section{Banspecifikation}
\label{sec:banspec}

\section{Systemskiss}
\label{sec:systemskiss}

\section{Projektplan}
\label{sec:projektplan}

\section{Tidplan}
\label{sec:tidplan}

\section{Designspecifikation}
\label{sec:designspec}

\section{Teknisk dokumentation}
\label{sec:tekdok}

\section{Sensoranalys}
\label{sec:sensoruppgift}

\section{Litiumjonbatterier och servomotorer}
\label{sec:batteriservouppgift}

\section{Linjeföljning och reglering av robotarm}
\label{sec:regleruppgift}




\end{document} 
