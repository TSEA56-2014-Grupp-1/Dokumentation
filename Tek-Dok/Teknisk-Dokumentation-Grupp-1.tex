\documentclass[a4paper,12pt]{article}
\usepackage{graphicx}

\usepackage{epstopdf}
\usepackage{gensymb}
\usepackage{float}
\usepackage{mathtools}
\usepackage{setspace}
\usepackage{tabularx}
\title{Teknisk Dokumentation}
\input{LIPS.tex}

\usepackage{sectsty}
\allsectionsfont{\sffamily}

\frenchspacing

\renewcommand{\thepage}{\roman{page}}

\newcommand{\LIPSartaltermin}{VT14}
\newcommand{\LIPSkursnamn}{TSEA56 Elektronik kandidatprojekt}

\newcommand{\LIPSprojekttitel}{Lagerrobot}

\newcommand{\LIPSprojektgrupp}{Grupp 1}
\newcommand{\LIPSgruppepost}{tsea56-2014-grupp-1@googlegroups.com}
\newcommand{\LIPSdokumentansvarig}{LIPS Teknisk Dokumentation}

\newcommand{\LIPSkund}{ISY, Linköpings universitet, 581\,83 Linköping}
\newcommand{\LIPSkundkontakt}{Tomas Svensson, 013-28 13 68, tomass@isy.liu.se}
\newcommand{\LIPSkursansvarig}{Tomas Svensson, 013-28 13 68, 3B:528, tomass@isy.liu.se}
\newcommand{\LIPShandledare}{Anders Nilsson, 3B:512, 013-28 26 35, anders.p.nilsson@liu.se}


\newcommand{\LIPSdokumenttyp}{Teknisk Dokumentation}
\newcommand{\LIPSredaktor}{Karl Linderhed}
\newcommand{\LIPSversion}{1.0}
\newcommand{\LIPSdatum}{\dagensdatum}

\newcommand{\LIPSgranskare}{}
\newcommand{\LIPSgranskatdatum}{}
\newcommand{\LIPSgodkannare}{}
\newcommand{\LIPSgodkantdatum}{}

\begin{document}

\LIPStitelsida

%% Argument till \LIPSgruppmedlem: namn, roll i gruppen, telefonnummer, epost
\begin{LIPSprojektidentitet}
  \LIPSgruppmedlem{Karl Linderhed}{Projektledare (PL)}{073-679 59 59}{karli315@student.liu.se}
  \LIPSgruppmedlem{Patrik Nyberg}{Dokumentansvarig (DOK)}{073 -049 59 90}{patny205@student.liu.se}
  \LIPSgruppmedlem{Johan Lind}{}{070-897 58 24}{johli887@student.liu.se}
  \LIPSgruppmedlem{Erik Nybom}{}{070-022 47 85}{eriny778@student.liu.se}
  \LIPSgruppmedlem{Andreas Runfalk}{}{070-564 23 79}{andru411@student.liu.se}
  \LIPSgruppmedlem{Philip Nilsson}{}{073-528 48 86}{phini326@student.liu.se}
  \LIPSgruppmedlem{Lucas Nilsson}{}{073-059 42 94}{lucni395@student.liu.se}
\end{LIPSprojektidentitet}


\renewcommand*\contentsname{Innehåll}
\begin{spacing}{0.5}
\tableofcontents{}
\end{spacing}
\newpage

%% Argument till \LIPSversionsinfo: versionsnummer, datum, ändringar, utfört av, granskat av
%\addcontentsline{toc}{section}{Dokumenthistorik}
\begin{LIPSdokumenthistorik}
  \LIPSversionsinfo{1.0}{}{}{}{}
\end{LIPSdokumenthistorik}
\newpage

\renewcommand{\thepage}{\arabic{page}}
\setcounter{page}{1}

\section{Inledning}
\emph{Bakgrund och syfte}


\section{Produktbeskrivning}
\emph{En bild på produkten och en beskrivning av hur den fungerar.
Beskriv vad den används till.}

\section{Teori}
\emph{Beskrivning av regleralgoritmer m.m.}

\section{Systemöversikt}
\emph{Ett översiktligt blockschema och en beskrivning av hela systemet.}

\section{Gemensamma funktioner}
Detta avsnitt beskriver olika funktioner som inbegriper eller används av flera olika moduler.

\subsection{Protokoll för seriell kommunikation mellan robot och PC}
\label{sec:bt-protokoll}

\subsection{USART}
\label{sec:usart}

\subsection{Intern buss}
\label{sec:bus}

\section{Sensorenhet}
\emph{Det här och de följande avsnitten innehåller mera detaljerade blockschemor och beskrivningar av varje modul.
Tänk på läsbarheten och växla mellan figurer och text.}
\section{Chassienhet}

\section{Armenhet}

\section{Kommunikationsenhet}
Kommunikationsenheten är robotens gränssnitt för interaktion med omvärlden och har två uppgifter. Den ska:
\begin{itemize}
\item agera som ett gränssnitt mot LCD-displayen och möjliggöra att alla delsystem kan skriva ut information på den.
\item hantera den seriella kommunikationen över Bluetooth och möjliggöra att roboten kan skicka och ta emot information till och från PC-gränssnittet.
\end{itemize}

\subsection{Seriell kommunikation över Bluetooth}
Anslutningen till datorn görs med bluetoothmodemet BlueSMiRF Gold som är monterat på roboten. På PC-sidan skapas vid parkopplingen med modemet en virtuell serieport, som emulerar en fysisk COM-port eller motsvarande. Kommunikationen sker sedan över denna port som om den vore en vanlig RS-232-port. 

På robotsidan används AVR-processorns inbyggda modul för UART, och det gemensamma biblioteket för USART som används av flera delsystem -- se avsnitt \ref{sec:usart}. Följande parametrar ställs in i det protokollet som roboten använder:
\begin{itemize}
\item Datahastigheten är 115 200 bps.
\item Data skickas som 8-bitars värden utan någon paritetsbit.
\item Ingen flödeshantering eller handskakning används.
\item Varje värde om 8 bitar avslutas med en stoppbit.
\end{itemize}

Hur olika typer av information överförs mellan robot och PC beskrivs närmare i avsnitt \ref{sec:bt-protokoll}.


\section{PC-gränssnitt}

\section{Slutsatser}
\emph{Vilka förbättringar skulle kunna göras?}

\newpage
\section*{Referenser}


\newpage
\appendix


\section{Kravspecifikation}
\label{sec:kravspec}

\section{Banspecifikation}
\label{sec:banspec}

\section{Systemskiss}
\label{sec:systemskiss}

\section{Projektplan}
\label{sec:projektplan}

\section{Tidplan}
\label{sec:tidplan}

\section{Designspecifikation}
\label{sec:designspec}

\section{Teknisk dokumentation}
\label{sec:tekdok}

\section{Sensoranalys}
\label{sec:sensoruppgift}

\section{Litiumjonbatterier och servomotorer}
\label{sec:batteriservouppgift}

\section{Linjeföljning och reglering av robotarm}
\label{sec:regleruppgift}




\end{document} 
