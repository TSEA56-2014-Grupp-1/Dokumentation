\section{Kommunikationsenhet}
Kommunikationsenheten är robotens gränssnitt för interaktion med omvärlden och har två uppgifter. Den ska:
\begin{itemize}
\item agera som ett gränssnitt mot LCD-displayen och möjliggöra att alla delsystem kan skriva ut information på den.
\item hantera den seriella kommunikationen över Bluetooth och möjliggöra att roboten kan skicka och ta emot information till och från PC-gränssnittet.
\end{itemize}

\subsection{Seriell kommunikation över Bluetooth}
Anslutningen till datorn görs med bluetoothmodemet \emph{BlueSMiRF Gold} som är monterat på roboten. På PC-sidan skapas vid parkoppling med modemet en virtuell serieport, som emulerar en fysisk COM-port eller motsvarande. Kommunikationen sker sedan över denna port som om den vore en vanlig RS-232-port. 

På robotsidan används AVR-processorns inbyggda modul för UART, och det gemensamma biblioteket för USART som används av flera delsystem -- se avsnitt \ref{sec:usart}. Följande parametrar ställs in i det protokollet som roboten använder:
\begin{itemize}
\item Datahastigheten är 115 200 bps.
\item Data skickas som 8-bitars värden utan någon paritetsbit.
\item Ingen flödeshantering eller handskakning används.
\item Varje värde om 8 bitar avslutas med en stoppbit.
\end{itemize}

Hur olika typer av information överförs mellan robot och PC beskrivs närmare i avsnitt \ref{sec:bt-protokoll}.
