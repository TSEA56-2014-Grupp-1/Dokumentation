\section{Kommunikationsenhet}
Kommunikationsenheten är robotens gränssnitt för interaktion med omvärlden och har två uppgifter. Den ska
\begin{itemize}
\item agera som ett gränssnitt mot LCD-skärmen och möjliggöra att alla delsystem kan skriva ut information på den.
\item hantera den seriella kommunikationen över blåtand och möjliggöra att roboten kan skicka och ta emot information till och från PC-gränssnittet.
\end{itemize}

Figur \ref{fig:kommblock} ger en övergripande bild över kommunikationsenhetens beståndsdelar.

\nyBild{Block-komm.png}{Blockschema över kommunikationsenheten.}{kommblock}{0.9}

\subsection{LCD-gränssnitt}
LCD-skärmen som är monterad på kommunikationsenheten är av modellen JM162A. Den styrs genom en parallell databuss och ett antal styrsignaler från kommunikationsenhetens processor till en intern processor i LCD-enheten. Skärmen har två rader med 16 tecken vardera.

Information skickas till LCD-enheten antingen som instruktioner eller data. Styrsignalernas koppling visas i figur \ref{fig:kommblock}. \verb|RS| styr om databussens värde tolkas som instruktion eller data, \verb|R/W| styr ifall information ska skrivas eller läsas, och styrsignalen \verb|E| aktiverar överföring av information. Genom att låta \verb|E| gå från hög till låg läser LCD-enheten in värdet som finns på databussen, antingen som en instruktion som ska utföras (om \verb|RS| är låg) eller som data som ska lagras i LCD-enhetens minne (om \verb|RS| är hög). Figur \ref{fig:lcdinit} visar vilka kommandon som skickas för att initiera LCD-enheten.

\nyBild{lcd-init.png}{Flödesschema över initialisering av LCD-enhet. Kommandon som skickas är kursiverade och har värdena på DB (i binär notation), RS och R/W som parametrar.}{lcdinit}{0.4}

Efter denna initiering kan data skrivas ut på skärmen genom att sätta \verb|RS| till 1 och skriva ut ASCII-koden\footnote{American Standard Code for Information Interchange, ett sätt att koda grundläggande alfanumeriska tecken och andra symboler.} för en symbol på databussen. Genom att först skicka en instruktion för att ställa in vilken adress i dataminnet som ska skrivas till härnäst kan man välja var en symbol ska skrivas ut, en viss adress i dataminnet svarar mot en position på skärmen.

Alla delsystem kan skriva ut information på skärmen genom ett gemensamt bibliotek, \textit{lcd\_interface}, som beskrivs närmare i avsnitt \ref{sec:lcd_interface}.


\subsection{Seriell kommunikation över blåtand}
Anslutningen till datorn sker med blåtandsmodemet \emph{BlueSMiRF Gold} som är monterat på roboten. På PC-sidan skapas vid parkoppling med modemet en virtuell serieport, som emulerar en fysisk COM-port eller motsvarande. Kommunikationen sker sedan över denna port som om den vore en vanlig RS-232-port. 

På robotsidan används AVR-processorns inbyggda modul för USART, och det gemensamma biblioteket för USART som används av flera delsystem -- se avsnitt \ref{sec:usart}. Följande parametrar ställs in i det protokoll som roboten använder:
\begin{itemize}
\item Datahastigheten är 115 200 bps.
\item Data skickas som 8-bitars värden utan någon paritetsbit.
\item Ingen flödeshantering eller handskakning används.
\item Varje värde om 8 bitar avslutas med en stoppbit.
\end{itemize}

Hur olika typer av information överförs mellan robot och PC beskrivs närmare i avsnitt \ref{sec:bt-protokoll}.
